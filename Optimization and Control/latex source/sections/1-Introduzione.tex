% filepath: c:\Users\mrheltic\Documents\GitHub\dispense\Optimization and Control\latex source\sections\1-Introduzione.tex

\chapter{Introduzione}
Il Corso affronterà gli aspetti tecnologici e metodologici della \textbf{Ricerca Operativa} (\textbf{Operations Research}). In particolare, cominciamo parlando delle sue fasi.

\section{Fasi del metodo della ricerca operativa}

\subsection{Analisi del Sistema e Formulazione del Problema}
\begin{itemize}
    \item Si comincia con l'osservazione dettagliata di un \textbf{sistema reale}, con l'obiettivo di isolare un \textbf{problema gestionale} specifico
    \item Si definiscono in modo rigoroso gli \textbf{obiettivi} dello studio, analizzando le caratteristiche del sistema
    \item Vengono identificati i \textbf{parametri principali} e le \textbf{variabili} del sistema, distinguendo tra quelle \textbf{controllabili} e quelle \textbf{non controllabili}
\end{itemize}

\subsection{Costruzione del modello}
\begin{itemize}
    \item Partendo dall'analisi precedente si costruisce un \textbf{modello rappresentativo} del problema
    \item In una prima fase il modello può essere \textbf{qualitativo} (cioè una descrizione informale del sistema), successivamente si sviluppa un \textbf{modello quantitativo} (ovvero una rappresentazione matematica del sistema reale), ne parleremo successivamente in modo più approfondito
    \item Un modello di ottimizzazione deve:
    \begin{itemize}
        \item Rappresentare il comportamento del sistema tramite \textbf{variabili decisionali}
        \item Definire una \textbf{funzione obiettivo} da massimizzare o minimizzare
        \item Includere eventuali \textbf{vincoli} che limitano le scelte possibili
    \end{itemize}
\end{itemize}

\subsection{Verifica del modello e Test delle Ipotesi}
\begin{itemize}
    \item Una volta costruito, il modello deve essere \textbf{verificato sperimentalmente}
    \item Prima ancora della sperimentazione è necessario effettuare una \textbf{raccolta dati} accurata
    \item Le sperimentazioni possono essere:
    \begin{itemize}
        \item \textbf{Manipolazione} del modello per ottenere informazioni rilevanti
        \item \textbf{Ciclo di feedback}, ovvero i risultati suggeriscono delle modifiche al modello, che vengono testate nelle iterazioni successive
    \end{itemize}
\end{itemize}

\subsection{Conclusioni e Implementazione}
\begin{itemize}
    \item L'ultima fase consiste nel trarre \textbf{conclusioni} dalla soluzione del modello e \textbf{implementarle} nel sistema reale
    \item È una fase \textbf{cruciale} in quanto i benefici dello studio si realizzano soltanto quando le soluzioni vengono effettivamente applicate
    \item Bisogna \textbf{comunicare in modo chiaro} le soluzioni ai decisori, affinché possano essere adottate
\end{itemize}

\section{Modelli Qualitativi}
Un \textbf{modello quantitativo} di ottimizzazione è generalmente composto da:
\begin{itemize}
    \item \textbf{Variabili decisionali}: Le scelte da compiere, la soluzione consiste nel determinare i valori ottimali di queste variabili
    \item \textbf{Funzioni obiettivo}: misura la qualità o la desiderabilità della soluzione, può essere un costo da minimizzare o un profitto da massimizzare
    \item \textbf{Vincoli}: limitano le scelte possibili, possono essere delle equazioni o disequazioni che riflettono i limiti tecnologici, economici o fisici
    \item \textbf{Parametri}: valori noti (costanti) che collegano le variabili decisionali alla funzione obiettivo e ai vincoli
\end{itemize}

\textbf{N.B.}: Non tutti i modelli devono necessariamente includere tutti e tre gli elementi fondamentali (variabili, vincoli o funzione obiettivo), ad esempio abbiamo:
\begin{itemize}
    \item Un modello senza vincoli è detto \textbf{ottimizzazione non vincolata}
    \item Un modello senza funzione obiettivo può essere usato per trovare una \textbf{soluzione fattibile}, senza ottimizzare nulla
\end{itemize}

\section{Osservazioni generali sui modelli}
Bisogna ricordare che tutti i modelli sono, per loro natura, delle \textbf{astrazioni della realtà}. Questo significa che la \textbf{soluzione ottimale} ottenuta dal modello potrebbe non coincidere perfettamente con la soluzione ottimale del problema reale.

Tuttavia, se il modello è \textbf{ben formulato} e se vengono effettuate \textbf{sperimentazioni} e \textbf{raffinamenti} prima di cercare la soluzione ottimale, è ragionevole aspettarsi che la soluzione ottenuta sia una \textbf{buona approssimazione} del problema reale.

\section{Algoritmi ed euristiche}
La soluzione di un modello può essere ottenuta tramite un \textbf{algoritmo}, ovvero:
\begin{itemize}
    \item Un insieme di \textbf{regole o procedure} da seguire in modo sequenziale o iterativo
    \item L'obiettivo è trovare o avvicinarsi alla \textbf{migliore soluzione possibile} per il modello dato
\end{itemize}

\textbf{N.B.} In alcuni casi si utilizzano le \textbf{euristiche}, cioè strategie semplificate per trovare \textbf{soluzioni accettabili} in tempi ragionevoli, soprattutto quando il problema è troppo complesso per essere risolto esattamente.

\section{Esempio di problema di pianificazione della produzione}
Un'azienda produce due prodotti, chiamati ``A'' e ``B''. Attualmente la produzione è organizzata in modo da alternare settimanalmente la produzione di ciascun prodotto.

Un giovane analista di ricerca operativa suggerisce che potrebbe essere più \textbf{efficiente} e \textbf{redditizio} produrre entrambi i prodotti ogni settimana, ottimizzando l'uso dei macchinari.

\subsection{Fase 1: Analisi di fattibilità}
Un team composto dall'analista e dai manager aziendali analizza la \textbf{fattibilità tecnica}, \textbf{operativa} ed \textbf{economica} di modificare il processo produttivo. La conclusione è che vale la pena tentare di ottimizzare il sistema.

\subsection{Fase 2: Costruzione del modello matematico}
\paragraph{\textbf{Variabili decisionali}}
\begin{itemize}
    \item $x_1$: Numero di lotti (da 1000 unità) del prodotto A da produrre in una settimana
    \item $x_2$: Numero di lotti (da 1000 unità) del prodotto B da produrre in una settimana
\end{itemize}

\paragraph{\textbf{Funzione obiettivo}}
L'obiettivo è \textbf{massimizzare} il profitto netto totale:
\begin{itemize}
    \item Profitto per lotto di A: 20\$
    \item Profitto per lotto di B: 50\$
\end{itemize}
Bisogna quindi massimizzare $P=20x_1+50x_2$

\paragraph{\textbf{Vincoli di produzione}}
I prodotti richiedono tempo su tre macchine, con le seguenti caratteristiche:

\begin{table}[h]
\centering
\begin{tabular}{|c|c|c|c|}
\hline
\textbf{Macchina} & \textbf{Ore per A} & \textbf{Ore per B} & \textbf{Capacità settimanale} \\
\hline
1 & 1 & 2 & 35 ore \\
\hline
2 & 2 & 1 & 40 ore \\
\hline
3 & 1 & 3 & 37 ore \\
\hline
\end{tabular}
\end{table}

Da questi dati si ottengono i vincoli:
\begin{equation}
\begin{cases}
x_1 + 2x_2 \leq 35 & \text{(Macchina 1)} \\
2x_1 + x_2 \leq 40 & \text{(Macchina 2)} \\
x_1 + 3x_2 \leq 37 & \text{(Macchina 3)} \\
x_1 \geq 0, x_2 \geq 0 & \text{(Non negatività)}
\end{cases}
\end{equation}

\paragraph{\textbf{Modello matematico completo}}
Volendo unire tutte le informazioni ottenute finora possiamo quindi scrivere il nostro modello completo:

\begin{align*}
\text{Massimizzare } & P=20x_1+50x_2 \\
\text{soggetto a: } & \\
& \begin{cases}
x_1+2x_2 \leq 35 \\
2x_1+x_2 \leq 40 \\
x_1+3x_2 \leq 37 \\
x_1 \geq 0, x_2 \geq 0
\end{cases}
\end{align*}

Questo è un modello di \textbf{programmazione lineare}, poiché sia la funzione obiettivo che i vincoli sono espressi in forma \textbf{lineare}.