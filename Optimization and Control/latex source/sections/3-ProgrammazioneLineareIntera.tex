\section{Programmazione Lineare Intera}

\subsection{Introduzione}
La \textbf{\textit{programmazione lineare intera (PLI)}} è una branca della \textbf{\textit{programmazione matematica}} che si occupa di problemi di ottimizzazione in cui 
\textbf{\textit{alcune o tutte le variabili decisionali sono vincolate a essere numeri interi}} (inclusi valori binari, che possono essere 0 o 1).

Tipologie di PLI:
\begin{itemize}
    \item \textbf{\textit{PI pura}}: tutte le variabili decisionali sono intere o binarie
    \item \textbf{\textit{PI mista}}: alcune variabili sono intere o binarie, altre sono continue
\end{itemize}

La PI è particolarmente utile per modellare problemi decisionali in cui:
\begin{itemize}
    \item Le variabili rappresentano \textbf{\textit{quantità indivisibili}} (es. un numero di macchine, persone o oggetti)
    \item È necessario scegliere tra un \textbf{\textit{insieme finito di alternative}} (es. accendere o spegnere una macchina, selezionare un percorso)
\end{itemize}

\underline{\textbf{NB.:}} In alcuni casi, il vincolo di interezza può essere rilassato se 
l'arrotondamento della soluzione continua non compromette significativamente:
\begin{itemize}
    \item Il rispetto dei vincoli del problema
    \item Il valore della funzione obiettivo
\end{itemize}

\subsection{Modelli di Taglio Ottimo}
I \textbf{\textit{modelli di taglio ottimo}} sono un'applicazione classica della PI. 
L'obiettivo è \textbf{\textit{minimizzare lo spreco (sfrido)}} derivante dal taglio dei 
materiali (come stoffa, pelle, lamiera, ecc.) in pezzi di dimensioni più piccole.

\subsection{Descrizione del Problema}
Il problema di taglio ottimo può essere descritto come segue:
\begin{itemize}
    \item Si parte da \textbf{\textit{moduli standard}} di lunghezza $D$.
    \item Si devono ottenere moduli più piccoli di lunghezza $d_i$ con $i = 1, \ldots, m$.
    \item Per ogni tipo $i$, si richiede una quantità $r_i$ di moduli di lunghezza $d_i$.
    \item Esistono $n$ \textbf{\textit{schemi di taglio}} diversi, ciascuno dei quali specifica come tagliare un modulo standard
\end{itemize}

I parametri del problema sono:
\begin{itemize}
    \item $D$: lunghezza del modulo standard
    \item $d_i$: lunghezza del modulo di tipo $i$
    \item $r_i$: quantità richiesta di moduli di tipo $i$
    \item $n$: numero di schemi di taglio disponibili
    \item $a_{ij}$: numero di moduli di tipo $i$ ottenuti dallo schema di taglio $j$
    \item $x_j$: numero di moduli standard tagliati secondo lo schema di taglio $j$
\end{itemize}

\subsection{Formulazione del Modello di Programmazione Lineare Intera}
\subsubsection{Funzione Obiettivo}
L'obiettivo è \textbf{\textit{minimizzare lo spreco totale di materiale}}, che può essere espresso come:
\[
\min z(x) = \sum_{j=1}^{n} x_j
\]

\subsubsection{Variabili decisionali}
\begin{itemize}
    \item $x_j \in \mathbb{Z}_{\geq 0}$: \textbf{\textit{numero di moduli standard tagliati secondo lo schema $j$}}.
\end{itemize}

\subsubsection{Formulazione matematica}
\[
\begin{aligned}
\min \quad & z(x) = \sum_{j=1}^{n} x_j \\
\text{soggetto a} \quad & \sum_{j=1}^{n} a_{ij} x_j \geq r_i \quad \text{per } i = 1, \dots, m \\
& x_j \in \mathbb{Z}_{\geq 0} \quad \text{(variabile intera non negativa)} \quad \text{per } j = 1, \dots, n
\end{aligned}
\]

\section{Problema di Taglio Ottimo: Esempio 1}
\subsection{Descrizione del problema}

Un'azienda produce fogli di lamiera di zinco di dimensioni 
$10 \times 1\, m^2$. Questi fogli vengono tagliati in larghezza fissa (1 metro), 
ma con lunghezze variabili per soddisfare la domanda mensile di diversi formati.

\subsection{Dati del problema}

La tabella seguente riporta:
\begin{itemize}
    \item Le lunghezze richieste
    \item La domanda mensile per ciascuna lunghezza
    \item I 5 schemi di taglio disponibili (ognuno specifica quante unità di ciascun tipo si ottengono da un foglio standard)
\end{itemize}

\begin{table}[htbp]
    \centering
    \caption{Schemi di taglio disponibili}
    \begin{tabular}{|c|c|c|c|c|c|c|}
        \hline
        \textbf{Lunghezza [m]} & \textbf{Domanda mensile} & \textbf{Schema 1} & \textbf{Schema 2} & \textbf{Schema 3} & \textbf{Schema 4} & \textbf{Schema 5} \\
        \hline
        3.0 & 23 & 2 & 0 & 0 & 1 & 1 \\
        2.5 & 34 & 1 & 2 & 0 & 1 & 0 \\
        2.0 & 28 & 0 & 1 & 0 & 2 & 0 \\
        4.0 & 20 & 0 & 0 & 2 & 0 & 1 \\
        1.5 & 35 & 1 & 2 & 1 & 0 & 2 \\
        \hline
    \end{tabular}
\end{table}

\subsection{Formulazione matematica}
\subsubsection{Variabili decisionali}
Sia $x_j$ il \textbf{\textit{numero di fogli standard tagliati secondo lo schema $j$}}, con $j = 1, \ldots, 5$.


\subsubsection{Funzione obiettivo}
L'obiettivo è \textbf{\textit{minimizzare il numero totale di fogli utilizzati}}:
\[
\min z(x) = x_1 + x_2 + x_3 + x_4 + x_5
\]


\subsubsection{Vincoli}
Tra i vincoli abbiamo inizialmente quelli di \textbf{\textit{domanda}}, che garantiscono che la quantità richiesta di ogni lunghezza sia soddisfatta:
\[
\begin{aligned}
2x_1 + x_4 + x_5 &\geq 23 \quad &\text{(lunghezza 3.0 m)} \\
x_1 + 2x_2 + x_4 &\geq 34 \quad &\text{(lunghezza 2.5 m)} \\
x_2 + 2x_4 &\geq 28 \quad &\text{(lunghezza 2.0 m)} \\
2x_3 + x_5 &\geq 20 \quad &\text{(lunghezza 4.0 m)} \\
x_1 + 2x_2 + x_3 + 2x_5 &\geq 35 \quad &\text{(lunghezza 1.5 m)}
\end{aligned}
\]

Successivamente, dobbiamo aggiungere i vincoli di \textbf{\textit{non negatività e interezza}}:
\[
x_j \in \mathbb{Z}_{\geq 0} \quad \text{per } j = 1, \dots, 5
\]

\subsection{Formulazione completa del modello}
\[
\begin{aligned}
\min \quad & z(x) = x_1 + x_2 + x_3 + x_4 + x_5 \\
\text{soggetto a} \quad & 2x_1 + x_4 + x_5 \geq 23 \quad \text{(lunghezza 3.0 m)} \\
& x_1 + 2x_2 + x_4 \geq 34 \quad \text{(lunghezza 2.5 m)} \\
& x_2 + 2x_4 \geq 28 \quad \text{(lunghezza 2.0 m)} \\
& 2x_3 + x_5 \geq 20 \quad \text{(lunghezza 4.0 m)} \\
& x_1 + 2x_2 + x_3 + 2x_5 \geq 35 \quad \text{(lunghezza 1.5 m)} \\
& x_j \in \mathbb{Z}_{\geq 0} \quad \text{per } j = 1, \dots, 5
\end{aligned}
\]

\section{Modello dello zaino (Knapsack Problem)}

Il \textbf{\textit{modello dello zaino}} è un classico problema di \textbf{\textit{Programmazione Lineare Intera Binaria}}, molto utilizzato per rappresentare situazioni decisionali 
in cui si devono selezionare elementi da un insieme, rispettando un 
vincolo di capacità.

\subsection{Descrizione del problema}
Dato un insieme di $n$ oggetti, ciascuno caratterizzato da:
\begin{itemize}
    \item Un \textbf{\textit{valore}} $c_j$ (es. profitto, utilità)
    \item Un \textbf{\textit{peso}} $p_j$ (es. spazio, costo)
\end{itemize}

e una \textbf{\textit{capacità massima}} $b$ dello zaino, l'obiettivo è \textbf{\textit{massimizzare il valore degli oggetti selezionati}}, senza superare la capacità disponibile.

\subsection{Formulazione matematica}
\subsubsection{Variabili decisionali}

Sia $x_j$ una \textbf{\textit{variabile binaria}} che indica se l'oggetto $j$ è selezionato ($x_j = 1$) o meno ($x_j = 0$).

\subsubsection{Funzione obiettivo}
L'obiettivo è \textbf{\textit{massimizzare il valore totale degli oggetti selezionati}}:
\[
\max z(x) = \sum_{j=1}^{n} c_j x_j
\]

\subsubsection{Vincoli}
Il vincolo principale è che il \textbf{\textit{peso totale degli oggetti selezionati non deve superare la capacità dello zaino}}:
\[
\sum_{j=1}^{n} p_j x_j \leq b
\]

Inoltre, le variabili devono essere \textbf{\textit{binarie}}:
\[
x_j \in \{0, 1\} \quad \text{per } j = 1, \ldots, n
\]

\subsection{Formulazione completa del modello}
\[
\begin{aligned}
\max \quad & z(x) = \sum_{j=1}^{n} c_j x_j \\
\text{soggetto a} \quad & \sum_{j=1}^{n} p_j x_j \leq b \\
& x_j \in \{0, 1\} \quad \text{per } j = 1, \dots, n
\end{aligned}
\]

\subsection{Ipotesi}
\begin{itemize}
    \item $c_j \geq 0$: i valori degli oggetti sono positivi
    \item $p_j \leq b$: ogni oggetto può essere inserito singolarmente nello zaino
    \item $\sum_{j=1}^{n} p_j > b$: non è possibile inserire tutti gli oggetti, altrimenti la soluzione ottima sarebbe banale
\end{itemize}

\section{Selezione di Scatole per il trasporto - Esempio 2}
\subsection{Descrizione del problema}

Un'azienda deve caricare un furgone con \textbf{\textit{capacità massima di 10 quintali}} 
(\textbf{\textit{1000 kg}}). Sono disponibili \textbf{\textit{12 scatole isotermiche}}, ciascuna con un peso e un valore 
economico. L'obiettivo è \textbf{\textit{massimizzare il valore totale delle scatole caricate}}, 
senza superare la capacità massima del furgone.

\subsection{Dati del problema}
\begin{table}[htbp]
    \centering
    \caption{Caratteristiche delle scatole isotermiche}
    \begin{tabular}{|c|c|c|}
        \hline
        \textbf{Tipologia} & \textbf{Peso [kg]} & \textbf{Valore [€]} \\
        \hline
        1 & 114 & 50 \\
        2 & 90 & 50 \\
        3 & 75 & 80 \\
        4 & 80 & 10 \\
        5 & 55 & 10 \\
        6 & 140 & 30 \\
        7 & 120 & 50 \\
        8 & 105 & 80 \\
        9 & 66 & 100 \\
        10 & 98 & 20 \\
        11 & 165 & 60 \\
        12 & 100 & 50 \\
        \hline
    \end{tabular}
\end{table}

\subsection{Formulazione matematica}

\[
\begin{aligned}
\max \quad & z(x) = 50x_1 + 50x_2 + 80x_3 + 10x_4 + 10x_5 + 30x_6 \\
& \quad + 50x_7 + 80x_8 + 100x_9 + 20x_{10} + 60x_{11} + 50x_{12} \\
\text{soggetto a} \quad & 1.14x_1 + 0.90x_2 + 0.75x_3 + 0.80x_4 + 0.55x_5 + 1.40x_6 \\
& \quad + 1.20x_7 + 1.05x_8 + 0.66x_9 + 0.98x_{10} + 1.65x_{11} + 1.00x_{12} \leq 10 \\
& x_j \in \{0, 1\} \quad \text{per } j = 1, \dots, 12
\end{aligned}
\]

