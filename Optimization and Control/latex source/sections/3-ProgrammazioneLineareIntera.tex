\chapter{Programmazione Lineare Intera}

\section{Introduzione}
La \textbf{\textit{programmazione lineare intera (PLI)}} è una branca della \textbf{\textit{programmazione matematica}} che si occupa di problemi di ottimizzazione in cui 
\textbf{\textit{alcune o tutte le variabili decisionali sono vincolate a essere numeri interi}} (inclusi valori binari, che possono essere 0 o 1).

Tipologie di PLI:
\begin{itemize}
    \item \textbf{\textit{PI pura}}: tutte le variabili decisionali sono intere o binarie
    \item \textbf{\textit{PI mista}}: alcune variabili sono intere o binarie, altre sono continue
\end{itemize}

La PI è particolarmente utile per modellare problemi decisionali in cui:
\begin{itemize}
    \item Le variabili rappresentano \textbf{\textit{quantità indivisibili}} (es. un numero di macchine, persone o oggetti)
    \item È necessario scegliere tra un \textbf{\textit{insieme finito di alternative}} (es. accendere o spegnere una macchina, selezionare un percorso)
\end{itemize}

\underline{\textbf{NB.:}} In alcuni casi, il vincolo di interezza può essere rilassato se 
l'arrotondamento della soluzione continua non compromette significativamente:
\begin{itemize}
    \item Il rispetto dei vincoli del problema
    \item Il valore della funzione obiettivo
\end{itemize}

\section{Modelli di Taglio Ottimo}
I \textbf{\textit{modelli di taglio ottimo}} sono un'applicazione classica della PI. 
L'obiettivo è \textbf{\textit{minimizzare lo spreco (sfrido)}} derivante dal taglio dei 
materiali (come stoffa, pelle, lamiera, ecc.) in pezzi di dimensioni più piccole.

\subsection{Descrizione del Problema}
Il problema di taglio ottimo può essere descritto come segue:
\begin{itemize}
    \item Si parte da \textbf{\textit{moduli standard}} di lunghezza $D$.
    \item Si devono ottenere moduli più piccoli di lunghezza $d_i$ con $i = 1, \ldots, m$.
    \item Per ogni tipo $i$, si richiede una quantità $r_i$ di moduli di lunghezza $d_i$.
    \item Esistono $n$ \textbf{\textit{schemi di taglio}} diversi, ciascuno dei quali specifica come tagliare un modulo standard
\end{itemize}

I parametri del problema sono:
\begin{itemize}
    \item $D$: lunghezza del modulo standard
    \item $d_i$: lunghezza del modulo di tipo $i$
    \item $r_i$: quantità richiesta di moduli di tipo $i$
    \item $n$: numero di schemi di taglio disponibili
    \item $a_{ij}$: numero di moduli di tipo $i$ ottenuti dallo schema di taglio $j$
    \item $x_j$: numero di moduli standard tagliati secondo lo schema di taglio $j$
\end{itemize}

\subsection{Formulazione del Modello di Programmazione Lineare Intera}
\subsubsection{Funzione Obiettivo}
L'obiettivo è \textbf{\textit{minimizzare lo spreco totale di materiale}}, che può essere espresso come:
\[
\min z(x) = \sum_{j=1}^{n} x_j
\]

\subsubsection{Variabili decisionali}
\begin{itemize}
    \item $x_j \in \mathbb{Z}_{\geq 0}$: \textbf{\textit{numero di moduli standard tagliati secondo lo schema $j$}}.
\end{itemize}

\subsubsection{Formulazione matematica}
\[
\begin{aligned}
\min \quad & z(x) = \sum_{j=1}^{n} x_j \\
\text{soggetto a} \quad & \sum_{j=1}^{n} a_{ij} x_j \geq r_i \quad \text{per } i = 1, \dots, m \\
& x_j \in \mathbb{Z}_{\geq 0} \quad \text{(variabile intera non negativa)} \quad \text{per } j = 1, \dots, n
\end{aligned}
\]

\section{Problema di Taglio Ottimo: Esempio 1}
\subsection{Descrizione del problema}

Un'azienda produce fogli di lamiera di zinco di dimensioni 
$10 \times 1\, m^2$. Questi fogli vengono tagliati in larghezza fissa (1 metro), 
ma con lunghezze variabili per soddisfare la domanda mensile di diversi formati.

\subsection{Dati del problema}

La tabella seguente riporta:
\begin{itemize}
    \item Le lunghezze richieste
    \item La domanda mensile per ciascuna lunghezza
    \item I 5 schemi di taglio disponibili (ognuno specifica quante unità di ciascun tipo si ottengono da un foglio standard)
\end{itemize}

\begin{table}[htbp]
    \centering
    \caption{Schemi di taglio disponibili}
    \begin{tabular}{|c|c|c|c|c|c|c|}
        \hline
        \textbf{Lunghezza [m]} & \textbf{Domanda mensile} & \textbf{Schema 1} & \textbf{Schema 2} & \textbf{Schema 3} & \textbf{Schema 4} & \textbf{Schema 5} \\
        \hline
        3.0 & 23 & 2 & 0 & 0 & 1 & 1 \\
        2.5 & 34 & 1 & 2 & 0 & 1 & 0 \\
        2.0 & 28 & 0 & 1 & 0 & 2 & 0 \\
        4.0 & 20 & 0 & 0 & 2 & 0 & 1 \\
        1.5 & 35 & 1 & 2 & 1 & 0 & 2 \\
        \hline
    \end{tabular}
\end{table}

\subsection{Formulazione matematica}
\subsubsection{Variabili decisionali}
Sia $x_j$ il \textbf{\textit{numero di fogli standard tagliati secondo lo schema $j$}}, con $j = 1, \ldots, 5$.


\subsubsection{Funzione obiettivo}
L'obiettivo è \textbf{\textit{minimizzare il numero totale di fogli utilizzati}}:
\[
\min z(x) = x_1 + x_2 + x_3 + x_4 + x_5
\]


\subsubsection{Vincoli}
Tra i vincoli abbiamo inizialmente quelli di \textbf{\textit{domanda}}, che garantiscono che la quantità richiesta di ogni lunghezza sia soddisfatta:
\[
\begin{aligned}
2x_1 + x_4 + x_5 &\geq 23 \quad &\text{(lunghezza 3.0 m)} \\
x_1 + 2x_2 + x_4 &\geq 34 \quad &\text{(lunghezza 2.5 m)} \\
x_2 + 2x_4 &\geq 28 \quad &\text{(lunghezza 2.0 m)} \\
2x_3 + x_5 &\geq 20 \quad &\text{(lunghezza 4.0 m)} \\
x_1 + 2x_2 + x_3 + 2x_5 &\geq 35 \quad &\text{(lunghezza 1.5 m)}
\end{aligned}
\]

Successivamente, dobbiamo aggiungere i vincoli di \textbf{\textit{non negatività e interezza}}:
\[
x_j \in \mathbb{Z}_{\geq 0} \quad \text{per } j = 1, \dots, 5
\]

\subsection{Formulazione completa del modello}
\[
\begin{aligned}
\min \quad & z(x) = x_1 + x_2 + x_3 + x_4 + x_5 \\
\text{soggetto a} \quad & 2x_1 + x_4 + x_5 \geq 23 \quad \text{(lunghezza 3.0 m)} \\
& x_1 + 2x_2 + x_4 \geq 34 \quad \text{(lunghezza 2.5 m)} \\
& x_2 + 2x_4 \geq 28 \quad \text{(lunghezza 2.0 m)} \\
& 2x_3 + x_5 \geq 20 \quad \text{(lunghezza 4.0 m)} \\
& x_1 + 2x_2 + x_3 + 2x_5 \geq 35 \quad \text{(lunghezza 1.5 m)} \\
& x_j \in \mathbb{Z}_{\geq 0} \quad \text{per } j = 1, \dots, 5
\end{aligned}
\]

\section{Modello dello zaino (Knapsack Problem)}

Il \textbf{\textit{modello dello zaino}} è un classico problema di \textbf{\textit{Programmazione Lineare Intera Binaria}}, molto utilizzato per rappresentare situazioni decisionali 
in cui si devono selezionare elementi da un insieme, rispettando un 
vincolo di capacità.

\subsection{Descrizione del problema}
Dato un insieme di $n$ oggetti, ciascuno caratterizzato da:
\begin{itemize}
    \item Un \textbf{\textit{valore}} $c_j$ (es. profitto, utilità)
    \item Un \textbf{\textit{peso}} $p_j$ (es. spazio, costo)
\end{itemize}

e una \textbf{\textit{capacità massima}} $b$ dello zaino, l'obiettivo è \textbf{\textit{massimizzare il valore degli oggetti selezionati}}, senza superare la capacità disponibile.

\subsection{Formulazione matematica}
\subsubsection{Variabili decisionali}

Sia $x_j$ una \textbf{\textit{variabile binaria}} che indica se l'oggetto $j$ è selezionato ($x_j = 1$) o meno ($x_j = 0$).

\subsubsection{Funzione obiettivo}
L'obiettivo è \textbf{\textit{massimizzare il valore totale degli oggetti selezionati}}:
\[
\max z(x) = \sum_{j=1}^{n} c_j x_j
\]

\subsubsection{Vincoli}
Il vincolo principale è che il \textbf{\textit{peso totale degli oggetti selezionati non deve superare la capacità dello zaino}}:
\[
\sum_{j=1}^{n} p_j x_j \leq b
\]

Inoltre, le variabili devono essere \textbf{\textit{binarie}}:
\[
x_j \in \{0, 1\} \quad \text{per } j = 1, \ldots, n
\]

\subsection{Formulazione completa del modello}
\[
\begin{aligned}
\max \quad & z(x) = \sum_{j=1}^{n} c_j x_j \\
\text{soggetto a} \quad & \sum_{j=1}^{n} p_j x_j \leq b \\
& x_j \in \{0, 1\} \quad \text{per } j = 1, \dots, n
\end{aligned}
\]

\subsection{Ipotesi}
\begin{itemize}
    \item $c_j \geq 0$: i valori degli oggetti sono positivi
    \item $p_j \leq b$: ogni oggetto può essere inserito singolarmente nello zaino
    \item $\sum_{j=1}^{n} p_j > b$: non è possibile inserire tutti gli oggetti, altrimenti la soluzione ottima sarebbe banale
\end{itemize}

\subsection{Selezione di Scatole per il trasporto - Esempio 2}
\subsubsection{Descrizione del problema}

Un'azienda deve caricare un furgone con \textbf{\textit{capacità massima di 10 quintali}} 
(\textbf{\textit{1000 kg}}). Sono disponibili \textbf{\textit{12 scatole isotermiche}}, ciascuna con un peso e un valore 
economico. L'obiettivo è \textbf{\textit{massimizzare il valore totale delle scatole caricate}}, 
senza superare la capacità massima del furgone.

\subsubsection{Dati del problema}
\begin{table}[htbp]
    \centering
    \caption{Caratteristiche delle scatole isotermiche}
    \begin{tabular}{|c|c|c|}
        \hline
        \textbf{Tipologia} & \textbf{Peso [kg]} & \textbf{Valore [€]} \\
        \hline
        1 & 114 & 50 \\
        2 & 90 & 50 \\
        3 & 75 & 80 \\
        4 & 80 & 10 \\
        5 & 55 & 10 \\
        6 & 140 & 30 \\
        7 & 120 & 50 \\
        8 & 105 & 80 \\
        9 & 66 & 100 \\
        10 & 98 & 20 \\
        11 & 165 & 60 \\
        12 & 100 & 50 \\
        \hline
    \end{tabular}
\end{table}

\subsubsection{Formulazione matematica}

\[
\begin{aligned}
\max \quad & z(x) = 50x_1 + 50x_2 + 80x_3 + 10x_4 + 10x_5 + 30x_6 \\
& \quad + 50x_7 + 80x_8 + 100x_9 + 20x_{10} + 60x_{11} + 50x_{12} \\
\text{soggetto a} \quad & 1.14x_1 + 0.90x_2 + 0.75x_3 + 0.80x_4 + 0.55x_5 + 1.40x_6 \\
& \quad + 1.20x_7 + 1.05x_8 + 0.66x_9 + 0.98x_{10} + 1.65x_{11} + 1.00x_{12} \leq 10 \\
& x_j \in \{0, 1\} \quad \text{per } j = 1, \dots, 12
\end{aligned}
\]

\subsection{Modello dello zaino - Pianificazione degli investimenti}
Il modello dello zaino può essere applicato anche a problemi di 
pianificazione degli investimenti, dove si dispone di un budget limitato 
e si devono selezionare progetti tra alternative indipendenti e competitive.

È possibile "mappare" concettualmente i concetti
\begin{itemize}
    \item Oggetti $\rightarrow$ Progetti di investimento
    \item Peso $\rightarrow$ Capitale richiesto
    \item Valore $\rightarrow$ Rendimento atteso
    \item Capacità dello zaino $\rightarrow$ Budget disponibile
\end{itemize}

\subsection{Problema dello zaino - Esempio 3}
\subsubsection{Descrizione del problema}
Una società dispone di 200.000 euro da investire in tre progetti: A, B, C. 
Ogni progetto ha un costo e un rendimento atteso:

\begin{table}[htbp]
    \centering
    \caption{Caratteristiche dei progetti di investimento}
    \begin{tabular}{|c|c|c|}
        \hline
        \textbf{Progetto} & \textbf{Costo [k€]} & \textbf{Ritorno atteso [k€]} \\
        \hline
        A & 90 & 100 \\
        B & 80 & 80 \\
        C & 70 & 75 \\
        \hline
    \end{tabular}
\end{table}

\subsubsection{Formulazione matematica}

\[
\begin{aligned}
\max \quad & z(x) = 100x_A + 80x_B + 75x_C \\
\text{soggetto a} \quad & 90x_A + 80x_B + 70x_C \leq 200 \\
& x_A, x_B, x_C \in \{0, 1\}
\end{aligned}
\]

\subsection{Modello dello zaino - Oggetti in quantità variabile}
In questa variante, ogni variabile decisionale rappresenta quanti oggetti 
di un certo tipo inserire nello zaino, con un vincolo di capacità.
Le variabili sono quindi intere non negative.

\subsubsection{Formulazione matematica}
\[
\begin{aligned}
\max \quad & z(x) = \sum_{j=1}^{n} c_j x_j \\
\text{soggetto a} \quad & \sum_{j=1}^{n} p_j x_j \leq b \\
& x_j \in \mathbb{Z}_{\geq 0} \quad \text{per } j = 1, \dots, n
\end{aligned}
\]

\subsection{Riempimento di Scatole con biscotti - Esempio 4}
\subsubsection{Descrizione del problema}
Nel reparto confezionamento di un'azienda, si devono riempire scatole da 120g con 
biscotti di 9 tipi diversi. Ogni tipo ha un peso e una percentuale di cioccolato pregiato (valore).

\subsubsection{Dati del problema}
\begin{table}[htbp]
    \centering
    \caption{Caratteristiche dei diversi tipi di biscotti}
    \begin{tabular}{|c|c|c|}
        \hline
        \textbf{Tipo} & \textbf{Peso [g]} & \textbf{\% Cioccolato} \\
        \hline
        1 & 20 & 32 \\
        2 & 25 & 58 \\
        3 & 26 & 65 \\
        4 & 21 & 18 \\
        5 & 36 & 28 \\
        6 & 28 & 55 \\
        7 & 28 & 40 \\
        8 & 27 & 40 \\
        9 & 23 & 25 \\
        \hline
    \end{tabular}
\end{table}

\subsubsection{Formulazione matematica}
\[
\begin{aligned}
\max \quad & z(x) = 32x_1 + 58x_2 + 65x_3 + 18x_4 + 28x_5 + 55x_6 + 40x_7 + 40x_8 + 25x_9 \\
\text{soggetto a} \quad & 20x_1 + 25x_2 + 26x_3 + 21x_4 + 36x_5 + 28x_6 + 28x_7 + 27x_8 + 23x_9 \leq 120 \\
& x_j \in \mathbb{Z}_{\geq 0} \quad \text{per } j = 1, \dots, 9
\end{aligned}
\]

\subsection{Modello dello zaino multidimensionale}
Il modello dello zaino può essere esteso per considerare più risorse 
(es. capitale, manodopera, tempo, materie prime) o più periodi temporali.

\subsubsection{Formulazione generale}
Siano:
\begin{itemize}
    \item $m$: numero di risorse o periodi
    \item $p_{ij}$: quantità della risorsa $i$ richiesta dall'oggetto $j$
    \item $b_i$: disponibilità della risorsa $i$
\end{itemize}

\[
\begin{aligned}
\max \quad & z(x) = \sum_{j=1}^{n} c_j x_j \\
\text{soggetto a} \quad & \sum_{j=1}^{n} p_{ij} x_j \leq b_i \quad \text{per } i = 1, \dots, m \\
& x_j \in \mathbb{Z}_{\geq 0} \quad \text{per } j = 1, \dots, n
\end{aligned}
\]

\subsection{Problema dello zaino multidimensionale - Esempio 5}
\subsubsection{Descrizione del problema}
Una compagnia aerea sta valutando la realizzazione di 3 progetti per l'acquisto 
di nuovi aerei, distribuiti su un orizzonte temporale di 4 mesi. Ogni 
progetto richiede un certo investimento mensile, mentre il budget disponibile 
varia di mese in mese.

\subsubsection{Dati del problema}
\begin{table}[htbp]
    \centering
    \caption{Budget mensile disponibile}
    \begin{tabular}{|c|c|}
        \hline
        \textbf{Mese} & \textbf{Budget disponibile [€]} \\
        \hline
        1 & 1.000.000 \\
        2 & 750.000 \\
        3 & 950.000 \\
        4 & 1.400.000 \\
        \hline
    \end{tabular}
\end{table}

Il profitto annuo atteso per ciascun progetto è:
\begin{table}[htbp]
    \centering
    \caption{Profitto annuo atteso per i progetti}
    \begin{tabular}{|c|c|}
        \hline
        \textbf{Progetto} & \textbf{Profitto annuo atteso [€]} \\
        \hline
        Progetto 1 & 480.000 \\
        Progetto 2 & 300.000 \\
        Progetto 3 & 250.000 \\
        \hline
    \end{tabular}
\end{table}

I costi mensili per ciascun progetto sono:
\begin{table}[htbp]
    \centering
    \caption{Costi mensili per ciascun progetto}
    \begin{tabular}{|c|c|c|c|}
        \hline
        \textbf{Mese} & \textbf{Progetto 1 [€]} & \textbf{Progetto 2 [€]} & \textbf{Progetto 3 [€]} \\
        \hline
        1 & 750.000 & 200.000 & 980.000 \\
        2 & 550.000 & 150.000 & 70.000 \\
        3 & 820.000 & 80.000 & 600.000 \\
        4 & 500.000 & 105.000 & 203.000 \\
        \hline
    \end{tabular}
\end{table}

\subsubsection{Formulazione matematica}
\[
\begin{aligned}
\max \quad & z(x)= 480x_1 + 300x_2 + 250x_3 \\
\text{soggetto a} \quad & 750x_1 + 200x_2 + 980x_3 \leq 1.000 \quad \text{(budget mese 1)} \\
& 550x_1 + 150x_2 + 70x_3 \leq 750 \quad \text{(budget mese 2)} \\
& 820x_1 + 80x_2 + 600x_3 \leq 950 \quad \text{(budget mese 3)} \\
& 500x_1 + 105x_2 + 203x_3 \leq 1.400 \quad \text{(budget mese 4)} \\
& x_1, x_2, x_3 \in \{0,1\}
\end{aligned}
\]

Questa formulazione matematica esprime l'obiettivo di massimizzare il profitto annuo atteso 
selezionando una combinazione ottimale dei tre progetti. I vincoli garantiscono che, per ciascun mese, 
la somma dei costi dei progetti selezionati non superi il budget disponibile. Le variabili binarie 
indicano se un progetto viene realizzato ($x_j = 1$) o meno ($x_j = 0$).

Si noti che i coefficienti nella funzione obiettivo rappresentano il profitto annuo atteso in migliaia 
di euro, e i coefficienti nei vincoli rappresentano i costi mensili in migliaia di euro.

\section{Modelli di ottimizzazione con Costi Fissi di Avviamento}
In molti problemi reali, come nella pianificazione della produzione, 
è necessario sostenere costi fissi di avviamento per attivare una certa attività 
(es. configurazione dei macchinari). Questi costi:
\begin{itemize}
    \item Non dipendono dalla quantità prodotta
    \item Si attivano solo se la produzione è effettivamente avviata ($geq 0$)
\end{itemize}

\subsection{Formulazione matematica}

Per ogni prodotto $j$, definiamo:
\begin{itemize}
    \item $c_j$: costo variabile unitario di produzione
    \item $f_j$: costo fisso di avviamento della produzione
    \item $x_j$: quantità prodotta del prodotto $j$
\end{itemize}

\subsubsection{Funzione obiettivo}
La funzione di costo per il prodotto $j$ è:
\[
z_j(x_j) =
\begin{cases}
f_j + c_j x_j, & \text{se } x_j > 0 \\
0, & \text{se } x_j = 0
\end{cases}
\]

Questa funzione è non lineare e presenta una discontinuità in $x_j = 0$.

\subsubsection{Linearizzazione del modello}
Per rendere il modello risolvibile con tecniche di Programmazione Lineare Intera Mista, 
si introduce una variabile intera binaria $y_j$ che indica se il prodotto $j$ è attivo ($y_j = 1$) o meno ($y_j = 0$).
\begin{itemize}
    \item $x_j = 1$ se $x_j > 0$ (produzione attiva)
    \item $x_j = 0$ se $y_j = 0$ (produzione non attiva)
\end{itemize}

E presenta una funzione obiettivo lineare:
\[
\min z(x, y) = \sum_{j=1}^{n} (f_j y_j + c_j x_j)
\]

\subsubsection{Vincoli}
Per garantire coerenza tra $x_j$ e $y_j$, si introducono i seguenti vincoli:
\[
x_j \leq M_j y_j \quad \text{per } j = 1, \dots, n
\]

dove $M_j$ è un limite superiore alla produzione del prodotto $j$.
NB. La minimizzazione della funzione obiettivo implica che, se $y_j = 0$, allora $x_j = 0$ 
per evitare costi fissi inutili.

\subsubsection{Formulazione completa del modello}
Si supponga che i prodotti $j=1, \ldots, n$ siano semilavorati utilizzati per produrre 
$m$ prodotti finiti. Siano:
\begin{itemize}
    \item $d_i$: domanda del prodotto finito $i$
    \item $a_{ij}$: quantità del semilavorato $j$ necessaria per produrre un'unità del prodotto finito $i$
    \item $M_j$: capacità massima di produzione del semilavorato $j$
\end{itemize}

Il modello PIM binario è:
\[
\begin{aligned}
\min \quad & z(x, y) = \sum_{j=1}^{n} (f_j y_j + c_j x_j) \\
\text{soggetto a} \quad & \sum_{j=1}^{n} a_{ij} x_j \geq d_i \quad \text{per } i = 1, \dots, m \\
& x_j \leq M_j y_j \quad \text{per } j = 1, \dots, n \\
& x_j \geq 0 \quad \text{per } j = 1, \dots, n \\
& y_j \in \{0, 1\} \quad \text{per } j = 1, \dots, n
\end{aligned}
\]

\subsection{Attivazione ottimale di condizionatori - Esempio 6}
\subsubsection{Descrizione del problema}
Un'azienda deve decidere quali tra quattro climatizzatori deve attivare 
per garantire un flusso d'aria sufficiente a coprire un volume di 3600 m³ con 
15 ricambi d'aria all'ora, ovvero:
\[
\text{Flusso d'aria totale} \geq 3600 \times 15 = 54000 \text{ m³/h}
\]

\subsubsection{Dati del problema}
Ogni climatizzatore ha:
\begin{itemize}
    \item Una portata d'aria oraria (in m³/h)
    \item Un costo fisso di attivazione (€/h)
    \item Un costo variabile di erogazione (€/(m³/h))
\end{itemize}

\begin{table}[htbp]
    \centering
    \caption{Caratteristiche dei climatizzatori}
    \begin{tabular}{|c|c|c|c|}
        \hline
        \textbf{Climatizzatore} & \textbf{Portata (m³/h)} & \textbf{Costo fisso (€/h)} & \textbf{Costo variabile (€/m³/h)} \\
        \hline
        1 & 44.000 & 500 & 2,42 \\
        2 & 48.000 & 450 & 2,64 \\
        3 & 36.000 & 600 & 1,98 \\
        4 & 51.000 & 580 & 2,80 \\
        \hline
    \end{tabular}
\end{table}

\subsubsection{Formulazione matematica}
\paragraph{Variabili decisionali}
Siano:
\begin{itemize}
    \item $x_j$: portata d'aria del climatizzatore $j$ (m³/h), con $j = 1, \ldots, 4$
    \item $y_j$: variabile binaria che indica se il climatizzatore $j$ è attivo ($y_j = 1$) o meno ($y_j = 0$)
\end{itemize}

\paragraph{Funzione obiettivo}
L'obiettivo è minimizzare il costo totale (fisso + variabile):
\[
\min z(x, y) = 2.42x_1 + 2.64x_2 + 1.98x_3 + 2.80x_4 + 500y_1 + 450y_2 + 600y_3 + 580y_4
\]

\paragraph{Vincoli}
Il vincolo principale è garantire che la portata d'aria totale sia sufficiente:
\[
x_1 + x_2 + x_3 + x_4 \geq 54.000
\]

Inoltre ci sono dei vincoli di attivazione:
\[
\begin{aligned}
x_1 &\leq 44.000 y_1 \\
x_2 &\leq 48.000 y_2 \\
x_3 &\leq 36.000 y_3 \\
x_4 &\leq 51.000 y_4
\end{aligned}
\]

Per ultimo i vincoli di non negatività e interezza:
\[
\begin{aligned}
x_j &\geq 0 \quad \text{per } j = 1, \dots, 4 \\
y_j &\in \{0, 1\} \quad \text{per } j = 1, \dots, 4
\end{aligned}
\]

\subsubsection{Formulazione completa del modello}
\[
\begin{aligned}
\min \quad & z(x, y) = 2.42x_1 + 2.64x_2 + 1.98x_3 + 2.80x_4 + 500y_1 + 450y_2 + 600y_3 + 580y_4 \\
\text{soggetto a} \quad & x_1 + x_2 + x_3 + x_4 \geq 54.000 \\
& x_1 \leq 44.000 y_1 \\
& x_2 \leq 48.000 y_2 \\
& x_3 \leq 36.000 y_3 \\
& x_4 \leq 51.000 y_4 \\
& x_j \geq 0 \quad \text{per } j = 1, \dots, 4 \\
& y_j \in \{0, 1\} \quad \text{per } j = 1, \dots, 4
\end{aligned}
\]

\section{Modelli di Localizzazione}
I modelli di localizzazione sono strumenti quantitativi fondamentali 
per la pianificazione territoriale di reti di servizio. L'obiettivo è 
determinare dove localizzare centri di servizio (es. impianti produttivi, 
centri di distribuzione, ospedali) in modo da:
\begin{itemize}
    \item Soddisfare la domanda distribuita su un'area geografica
    \item Miminizzare una funzione di costo (es. costi di trasporto, operativi)
\end{itemize}

Queste decisioni di localizzazione sono:
\begin{itemize}
    \item Stragegiche e a lungo termine
    \item Implicano investimenti significativi
    \item Influenzano la struttura logistica e operativa dell'intera organizzazione
\end{itemize}

\section{Modello CPL - Capacited Plant Location}
Ci concentriamo sul modello CPL (Capacited Plant Location), una delle 
formulazioni più note e diffuse. In base alle ipotesi possiamo avere:
\begin{itemize}
    \item Monotipo: tutti i centri da localizzare sono dello stesso tipo
    \item Discreto: l'insieme dei siti potenziali $I$ è noto e finito
    \item Singolo prodotto: si considera un solo tipo di prodotto o servizio
    \item Un livello: i flussi sono solo tra i siti da localizzare e i nodi successivi (clienti o punti vendita)
\end{itemize}

\subsection{Formulazione matematica}
Il problema può essere rappresentato come un grafo bipartito orientato 
$G = (I \cup J, A)$, dove:
\begin{itemize}
    \item $I$: insieme dei siti potenziali (impianti), indicato nelle slide con $N_1$
    \item $J$: insieme dei nodi successivi (clienti o punti vendita), indicato nelle slide con $N_2$
    \item $A = I \times J$: insieme degli archi che collegano i siti ai clienti, indicato nelle slide con $s_{ij}$
\end{itemize}

Tra i parametri del problema abbiamo:
\begin{itemize}
    \item $k_{ij}$: costo di trasporto unitario dal sito $i$ a $j$
    \item $f_i$: costo fisso medio di attivazione del sito $i$ (ottenuto come media dei costi di attivazione dei siti)
    \item $d_j$: domanda del cliente $j \in J$
    \item $q_i$: capacità massima del sito $i \in I$
\end{itemize}

\subsubsection{Variabili decisionali}
Siano:
\begin{itemize}
    \item $x_{ij} \geq 0$: quantità trasportata dal sito $i$ al nodo $j$
    \item $y_i \in \{0,1\}$: variabile binaria che indica se il sito $i$ è attivo ($y_i = 1$) o meno ($y_i = 0$)
\end{itemize}

\subsubsection{Funzione obiettivo}
L'obiettivo è minimizzare il costo totale di trasporto e i costi fissi di attivazione:
\[
\min z(x, y) = \sum_{i \in I} f_i y_i + \sum_{(i,j) \in A} c_{ij} x_{ij}
\]

\subsubsection{Formulazione matematica completa}
\[
\begin{aligned}
\min \quad & z(s, y) = \sum_{i \in I}\sum_{j \in J} k_{ij} s_{ij} + \sum_{i \in I} f_i y_i\\
\text{soggetto a} \quad & \sum_{i \in I} s_{ij} = (\geq) d_j \quad \text{per } j \in J \\
& \sum_{j \in J} s_{ij} \leq q_i y_i \quad \text{per } i \in I \\
& s_{ij} \geq 0 \quad \text{per } i \in I, j \in J \\
& y_i \in \{0,1\} \quad \text{per } i \in I
\end{aligned}
\]

È possibile sostituire le variabili $s_{ij}$ con $d_{j} x{ij}$ dove $x_{ij}$ è la frazione della quantità media richiesta dal cliente $j$ che proviene dal sito $i$.
\[
\begin{aligned}
\min \quad & z(x, y) = \sum_{i \in I}\sum_{j \in J} k_{ij} d_j x_{ij} + \sum_{i \in I} f_i y_i  \\
\text{soggetto a} \quad & \sum_{i \in I} x_{ij} = 1 \quad \text{per } j \in J \\
& \sum_{j \in J} d_j x_{ij} \leq q_i y_i \quad \text{per } i \in I \\
& x_{ij} \geq 0 \quad \text{per } i \in I, j \in J \\
& y_i \in \{0,1\} \quad \text{per } i \in I
\end{aligned}
\]

\subsection{CPL - Esempio 7}
Un'azienda agricola deve decidere dove localizzare i propri impianti 
per sodisfare una certa richiesta media quotidiana di foraggio. 
Sono stati individuati 6 potenziali siti. Sono noti anche o costi fissi per 
ciascun silo potenziale, il costo giornaliero medio di stoccaggio per ogni quintale e 
il costo di trasportato per ogni quintale.

\subsubsection{Dati del problema}

La richiesta media quotidiana è:
\begin{table}[htbp]
    \centering
    \caption{Richiesta media quotidiana di foraggio}
    \begin{tabular}{|c|c|}
        \hline
        \textbf{Cliente} & \textbf{Richiesta media quotidiana [q]} \\
        \hline
        1 & 36 \\
        2 & 42 \\
        3 & 34 \\
        4 & 50 \\
        5 & 27 \\
        6 & 30 \\
        7 & 43 \\
        \hline
    \end{tabular}
\end{table}

\begin{table}[htbp]
    \centering
    \caption{Capacità massima dei siti potenziali}
    \begin{tabular}{|c|c|}
        \hline
        \textbf{Sito potenziale} & \textbf{Capacità massima giornaliera [q]} \\
        \hline
        1 & 80 \\
        2 & 90 \\
        3 & 110 \\
        4 & 120 \\
        5 & 100 \\
        6 & 120 \\
        \hline
    \end{tabular}
\end{table}

\begin{table}[htbp]
    \centering
    \caption{Costi fissi stimati per i siti potenziali}
    \begin{tabular}{|c|c|}
        \hline
        \textbf{Sito potenziale} & \textbf{Costo fisso stimato [€]} \\
        \hline
        1 & 321.420 \\
        2 & 350.640 \\
        3 & 379.860 \\
        4 & 401.775 \\
        5 & 350.640 \\
        6 & 336.030 \\
        \hline
    \end{tabular}
\end{table}

\begin{table}[htbp]
    \centering
    \caption{Costi giornalieri medi di stoccaggio}
    \begin{tabular}{|c|c|}
        \hline
        \textbf{Sito potenziale} & \textbf{Costo giornaliero medio di stoccaggio [€/q]} \\
        \hline
        1 & 0.15 \\
        2 & 0.18 \\
        3 & 0.20 \\
        4 & 0.18 \\
        5 & 0.15 \\
        6 & 0.17 \\
        \hline
    \end{tabular}
\end{table}

I costi di trasporto per ogni quintale di prodotto per ogni chilometro 
sono di 0.06 euro

\begin{table}[htbp]
    \centering
    \caption{Distanze in km tra i silos potenziali e i clienti}
    \begin{tabular}{|c|c|c|c|c|c|c|c|}
        \hline
        \textbf{Silos potenziali} & \multicolumn{7}{c|}{\textbf{Clienti}} \\
        \cline{2-8}
        & \textbf{1} & \textbf{2} & \textbf{3} & \textbf{4} & \textbf{5} & \textbf{6} & \textbf{7} \\
        \hline
        1 & 18 & 23 & 19 & 21 & 24 & 17 & 9 \\
        2 & 21 & 18 & 17 & 23 & 11 & 18 & 20 \\
        3 & 27 & 18 & 17 & 20 & 23 & 9 & 18 \\
        4 & 16 & 23 & 9 & 31 & 21 & 23 & 10 \\
        5 & 31 & 20 & 18 & 19 & 10 & 17 & 18 \\
        6 & 18 & 17 & 29 & 21 & 22 & 18 & 8 \\
        \hline
    \end{tabular}
\end{table}

\subsubsection{Formulazione matematica}

\paragraph{Variabili decisionali}
Siano:
\begin{itemize}
    \item $x_{ij}$: quantità di foraggio trasportata dal sito $i$ al cliente $j$ (in quintali)
    \item $y_i$: variabile binaria che indica se il sito $i$ è attivo ($y_i = 1$) o meno ($y_i = 0$)
\end{itemize}

\paragraph{Funzione obiettivo}
L'obiettivo è minimizzare il costo totale di trasporto e i costi fissi di attivazione:
\[
\min z(x, y) = \sum_{i=1}^{6} f_i y_i + \sum_{i=1}^{6} \sum_{j=1}^{7} k_{ij} x_{ij}
\]

Con:
\begin{itemize}
    \item $k_{ij}$: costo di trasporto unitario dal sito $i$ al cliente $j$, calcolato come:
    \[
    k_{ij} = 0.06 \times d_{ij}
    \]
    \item $d_{j}$: richiesta media quotidiana del cliente $j$
    \item $q_{i}$: capacità massima del sito $i$
    \item $f_i$: costo fisso di attivazione del sito $i$, calcolato come:
    \[
    321.420 / (365 \times 4) = 321.420/1461 = 220
    \] 
    dove 4 sono gli anni e 365 sono i giorni dell'anno. Questo va fatto per tutti i siti.
\end{itemize}

\subsection{Vincoli sull'attivazione dei siti}

Per ogni sito $i \in I$, si impongono soglie di attività:
\begin{itemize}
    \item $q_i^-$: soglia minima di attività (quantità minima che giustifica l'attivazione del sito)
    \item $q_i^+$: capacità massima del sito
\end{itemize}

Pertanto, i vincoli corrisponderebbero a:
\[
\begin{aligned}
\sum_{j \in J} d_j x_{ij} &\leq q_i^+ y_i \quad \forall i \in I \\
\sum_{j \in J} d_j x_{ij} &\geq q_i^- y_i \quad \forall i \in I
\end{aligned}
\]

Questo perché potrebbe essere non conveniente attivare un sito se la domanda è inferiore alla soglia minima di attività o 
superiore alla capacità massima del sito.

\subsection{CPL - Esempio 8}
Riferendoci al problema di localizzazione dei silos, assumiamo che 
il sesto silos, se acquisito deve avere un livello minimo di attività 
pari a 90 quintali e massimo di 120 quintali.

Quindi aggiungeremo il vincolo:
\[
36x_{61} + 42x_{62} + 34x_{63} + 50x_{64} + 27x_{65} + 30x_{66} + 43x_{67} \geq 90 y_6
\]

\subsection{SPL - Single Product Location}
Nel modello SPL (Single Product Location), si rimuovono i vincoli di capacità 
e si impone solo che un sito sia attivo se fornisce a qualche cliente:
\[
\sum_{j \in J} x_{ij} \leq |J| \cdot y_i \quad \forall i \in I
\]

\subsection{SPL - Esempio 9}
Nel caso in cui ogni silo possa soddisfare l'intera domanda, si sostituiscono 
i vincoli di capacità con:
\[
\begin{aligned}
x_{11} + x_{12} + x_{13} + x_{14} + x_{15} + x_{16} + x_{17} &\leq 7 y_1 \\
x_{21} + x_{22} + x_{23} + x_{24} + x_{25} + x_{26} + x_{27} &\leq 7 y_2 \\
x_{31} + x_{32} + x_{33} + x_{34} + x_{35} + x_{36} + x_{37} &\leq 7 y_3 \\
x_{41} + x_{42} + x_{43} + x_{44} + x_{45} + x_{46} + x_{47} &\leq 7 y_4 \\
x_{51} + x_{52} + x_{53} + x_{54} + x_{55} + x_{56} + x_{57} &\leq 7 y_5 \\
x_{61} + x_{62} + x_{63} + x_{64} + x_{65} + x_{66} + x_{67} &\leq 7 y_6
\end{aligned}
\]

