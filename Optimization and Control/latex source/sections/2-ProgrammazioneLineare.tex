\chapter{Programmazione Lineare}
\section{Problemi e Istanze}
Un problema P è una domanda o una questione la cui risposta dipende dal valore assunto da un insieme di parametri e variabili.
Un problema è formalmente definito da:
\begin{itemize}
    \item Descrizione dei parametri, ovvero le quantità note che caratterizzano il problema
    \item Descrizione delle proprietà che deve soddisfare la soluzione cercata
\end{itemize}

Una formulazione comune di un problema prevede:

\begin{itemize}
    \item L'insieme delle soluzioni ammissibili F, cioè tutte le possibili risposte che rispettano i vincoli imposti
    \item Una funzione obiettivo z che assegna un valore (costo, profitto, ecc.) a ciascuna soluzione
\end{itemize}

Formalmente:
\[
z \colon F \rightarrow \mathbb{R}
\]

Un'istanza di un problema P, invece, è una versione specifica del problema ottenuta fissando i valori dei parametri. In altre parole, è un caso concreto del problema generale.

\section{Problemi di Ottimizzazione}
Un problema di ottimizzazione è un problema in cui si cerca, tra tutte le soluzioni ammissibili $x \in F$, una soluzione ottima $x^*$ che massimizza o minimizza la funzione obiettivo $z$.

Formulazioni equivalenti:

\begin{itemize}

    \item Minimizzazione
    \[
    P = (F, z; \min) \Leftrightarrow \min\{z(x) : x \in F\}
    \]
    \item Massimizzazione
    \[
    P = (F, z; \max) \Leftrightarrow \max\{z(x) : x \in F\}
    \]
\end{itemize}

\section{Problemi di programmazione Lineare}
Un problema di programmazione lineare è un caso particolare di problema di ottimizzazione in cui:

\begin{itemize}
    \item La funzione obiettivo è lineare
    \item I vincoli che definiscono l'insieme F sono lineari e chiusi (cioè includono i punti di frontiera)
    \item Le variabili sono continue (non intere)
    \item I parametri sono deterministici (noti e non soggetti a incertezza)
\end{itemize}

La sua forma esplicita è del tipo:
\begin{align}
\max & \quad c_1x_1 + c_2x_2 + \cdots + c_nx_n \\
\text{soggetto a} & \quad a_{11}x_1 + a_{12}x_2 + \cdots + a_{1n}x_n \leq b_1 \\
& \quad \vdots \\
& \quad a_{m1}x_1 + a_{m2}x_2 + \cdots + a_{mn}x_n \leq b_m \\
& \quad x_1, x_2, \ldots, x_n \geq 0
\end{align}

È possibile scriverla anche in forma matriciale, ovvero:
\begin{align}
\max & \quad z = c^T x \\
\text{soggetto a} & \quad Ax \leq b \\
& \quad x \geq 0
\end{align}

Dove:


\begin{itemize}
    \item $c \in \mathbb{R}^n$ : vettore dei coefficienti della funzione obiettivo;
    \item $A \in \mathbb{R}^{m \times n}$ : matrice dei coefficienti dei vincoli (matrice tecnologica);
    \item $b \in \mathbb{R}^m$ : vettore dei termini noti;
    \item $x \in \mathbb{R}^n$ : vettore delle variabili decisionali;
    \item $F = \{x \in \mathbb{R}^n : Ax \leq b, x \geq 0\}$ : insieme delle soluzioni ammissibili.
\end{itemize}

\section{Esempio di pianificazione della produzione - 1}
\subsection{Descrizione}

L'obiettivo è massimizzare il profitto derivante dalla produzione e vendita di due tipi di biocarburanti, quello di tipo 1 e quello di tipo 2.

\subsection{Vincoli}
\begin{itemize}
    \item Produzione divisa in tre stabilimenti: uno per la preparazione, uno per la purificazione e uno per l'estrazione
    \item Ogni stabilimento ha una capacità giornaliera limitata (in termini di ore)
    \item Entrambi i biocarburanti devono essere lavorati in ciascuno dei tre stabilimenti
\end{itemize}
\subsection{Dati tecnologici}
\begin{table}[h]
\centering
\resizebox{\textwidth}{!}{
\begin{tabular}{|c|c|c|c|}
\hline
\textbf{Stabilimento} & \textbf{Biocarburante 1 (h/ton)} & \textbf{Biocarburante 2 (h/ton)} & \textbf{Capacità giornaliera (h)} \\
\hline
Preparazione   & 0.72 & 0.85 & 18 \\
\hline
Purificazione  & 1.68 & 1.42 & 18 \\
\hline
Estrazione     & 1.92 & 2.12 & 16 \\
\hline
\end{tabular}
}
\end{table}

\subsection{Formulazione matematica}
\subsubsection{Variabili decisionali}
Siano:

\begin{itemize}
    \item $x_1$ : quantità (in tonnellate) di Biocarburante 1 da produrre
    \item $x_2$ : quantità (in tonnellate) di Biocarburante 2 da produrre
\end{itemize}

Le variabili sono continue e non negative:

\[
x_1, x_2 \in \mathbb{R}_{\geq 0}
\]

\subsubsection{Funzione obiettivo}

Massimizzare il profitto totale:
\[
\max z(x) = 540x_1 + 590x_2
\]

\subsubsection{Vincoli di capacità}
\[
\begin{cases}
0.72x_1 + 0.85x_2 \leq 18 \quad &\text{(Preparazione)} \\
1.68x_1 + 1.42x_2 \leq 18 \quad &\text{(Purificazione)} \\
1.92x_1 + 2.12x_2 \leq 16 \quad &\text{(Estrazione)} \\
x_1, x_2 \geq 0 \quad &\text{(Non negatività)}
\end{cases}
\]

\section{Esempio di pianificazione della produzione - 2}
\subsection{Descrizione}
L'obiettivo è massimizzare il profitto settimanale, calcolato come differenza tra ricavi di vendita e costi di produzione, per tre tipi di argilla: A, B e C.

\subsection{Vincoli}
\begin{itemize}
    \item La produzione avviene in tre stabilimenti, ciascuno con una capacità settimanale limitata (in ore)
    \item Ogni tipo di argilla può essere lavorato in uno qualsiasi dei tre stabilimenti
    \item La quantità totale di argilla A deve essere compresa tra il 50\% e il 70\% della produzione complessiva
\end{itemize}

\subsection{Dati tecnologici}
\begin{table}[h]
\centering
\begin{tabular}{|c|c|c|c|c|c|}
\hline
\textbf{Stabilimento} & \textbf{Argilla A} & \textbf{Argilla B} & \textbf{Argilla C} & \textbf{Capacità} & \textbf{Costo} \\
& \textbf{18€/quintale} & \textbf{21€/quintale} & \textbf{24€/quintale} & \textbf{settimanale (h)} & \textbf{[€/h]} \\
\hline
1 & 0.18 & 0.21 & 0.24 & 90 & 3.52 \\
\hline
2 & 0.20 & 0.18 & 0.21 & 85 & 4.18 \\
\hline
3 & 0.12 & 0.22 & 0.23 & 80 & 3.98 \\
\hline
\end{tabular}
\end{table}

\subsection{Formulazione matematica}
\subsubsection{Variabili decisionali}

Siano:
\begin{itemize}
    \item $x_{iA}$ : quantità di argilla A prodotta nello stabilimento $i$
    \item $x_{iB}$ : quantità di argilla B prodotta nello stabilimento $i$
    \item $x_{iC}$ : quantità di argilla C prodotta nello stabilimento $i$
\end{itemize}

Quindi, siano:
\[
x_{iA}, x_{iB}, x_{iC} \in \mathbb{R}_{\geq 0}, \quad \text{per } i=1,2,3
\]

\subsubsection{Funzione obiettivo}
Massimizzare il profitto, ovvero la differenza tra ricavi e costi, cioè:
\begin{align*}
\max z = & \ 18(x_{1A} + x_{2A} + x_{3A}) + 21(x_{1B} + x_{2B} + x_{3B}) + 24(x_{1C} + x_{2C} + x_{3C}) \\
& - \left[ 3.52(0.18x_{1A} + 0.21x_{1B} + 0.24x_{1C}) \right. \\
& \left. + 4.18(0.20x_{2A} + 0.18x_{2B} + 0.21x_{2C}) \right. \\
& \left. + 3.98(0.12x_{3A} + 0.22x_{3B} + 0.23x_{3C}) \right]
\end{align*}

\subsubsection{Vincoli di capacità}
\begin{align}
\begin{cases}
    0.18x_{1A}+0.21x_{1B}+0.24x_{1C} \leq 90 \\
    0.20x_{2A}+0.18x_{2B}+0.21x_{2C} \leq 85 \\
    0.12x_{3A}+0.22x_{3B}+0.23x_{3C} \leq 80
\end{cases}
\end{align}

\subsubsection{Vincoli di proporzione sulla produzione di argilla A}
\begin{align}
x_{1A}+x_{2A}+x_{3A} &\geq 0.5 \cdot \sum_{i=1}^{3}(x_{iA}+x_{iB}+x_{iC}) \\
x_{1A}+x_{2A}+x_{3A} &\leq 0.7 \cdot \sum_{i=1}^{3}(x_{iA}+x_{iB}+x_{iC})
\end{align}

Dove: $T = \sum_{i=1}^{3}(x_{iA}+x_{iB}+x_{iC})$

\section{Mix di produzione - Esempio 3}

\subsection{Descrizione}
Si considera un problema di pianificazione della produzione in cui:
\begin{itemize}
    \item Sono presenti n prodotti, ciascuno dei quali richiede una certa quantità di m materie prime
    \item Le materie prime sono disponibili in quantità limitata
    \item Ogni prodotto genera un profitto unitario
\end{itemize}

\subsection{Obiettivo}
Determinare le quantità ottimali da produrre per massimizzare il profitto totale, rispettando i vincoli di disponibilità delle materie prime.

\subsection{Dati tecnologici}
\begin{table}[h]
\centering
\begin{tabular}{|l|c|c|c|c|c|c|}

\hline
Prodotto      & P1 & P2 & P3 & P4 & P5 & P6 \\ \hline
Profitto/unità & 30 & 45 & 24 & 26 & 24 & 33 \\ \hline
\end{tabular}
\caption{Profitto unitario per ogni prodotto}
\end{table}

\begin{table}[h]
\centering
\begin{tabular}{|l|c|c|c|c|c|c|c|}

\hline
Materia Prima & P1 & P2 & P3 & P4 & P5 & P6 & Disponibilità \\ \hline
Acciaio     & 1  & 4  & 0  & 4  & 2  & 0  & $\leq 1200$ \\ \hline
Legno       & 4  & 5  & 3  & 0  & 1  & 0  & $\leq 1160$ \\ \hline
Plastica    & 0  & 3  & 8  & 0  & 1  & 0  & $\leq 1780$ \\ \hline
Gomma       & 2  & 0  & 1  & 2  & 1  & 5  & $\leq 1050$ \\ \hline
Vetro       & 2  & 4  & 2  & 2  & 2  & 4  & $\leq 1360$ \\ \hline
\end{tabular}
\caption{Consumo di materie prime per ogni prodotto}
\end{table}

\subsection{Formulazione matematica}

\subsubsection{Variabili decisionali}
\[ 
x_i \in \mathbb{R}_{\geq 0}, \quad \text{per } i = 1, \dots, n
\]

\subsubsection{Funzione obiettivo}
\[ 
\max \sum_{i=1}^{n} c_i x_i
\]

\subsubsection{Vincoli}
\[ 
\sum_{i=1}^{n} a_{ji} x_i \leq b_j,\quad j = 1, \dots, m,\qquad x_i \geq 0, \quad i = 1, \dots, n.
\]

L'istanza specifica diventa:
\begin{align}
\max \quad & 30x_1 + 45x_2 + 24x_3 + 26x_4 + 24x_5 + 33x_6 \\[1mm]
\text{soggetto a} \quad
& x_1 + 4x_2 + 4x_4 + 2x_5 \leq 1200 \quad && \text{(Acciaio)} \\
& 4x_1 + 5x_2 + 3x_3 + x_5 \leq 1160 \quad && \text{(Legno)} \\
& 3x_2 + 8x_3 + x_5 \leq 1780 \quad && \text{(Plastica)} \\
& 2x_1 + x_3 + 2x_4 + x_5 + 5x_6 \leq 1050 \quad && \text{(Gomma)} \\
& 2x_1 + 4x_2 + 2x_3 + 2x_4 + 2x_5 + 4x_6 \leq 1360 \quad && \text{(Vetro)} \\
& x_1,\, x_2,\, x_3,\, x_4,\, x_5,\, x_6 \geq 0.
\end{align}

\section{\textbf{\textit{Problema della dieta - Esempio 4}}}

\subsection{\textbf{\textit{Descrizione}}}
Determinare le quantità ottimali di diversi alimenti per una dieta giornaliera che:
\begin{itemize}
    \item Soddisfi i requisiti nutrizionali minimi
    \item Minimizzi il costo totale
\end{itemize}

\subsection{Dati del problema}
\begin{table}[h]
\centering
\begin{tabular}{|c|c|c|c|c|}

\hline
Alimento & G1 & G2 & G3 & G4 \\ \hline
Costo/unità & 35 & 50 & 80 & 95 \\ \hline
\end{tabular}
\caption{Costo unitario per alimento}
\end{table}

\begin{table}[h]
\centering
\begin{tabular}{|c|c|c|c|c|c|}

\hline
Nutriente & G1 & G2 & G3 & G4 & Fabbisogno minimo \\
\hline
A & 2.2 & 3.4 & 7.2 & 1.5 & $\geq 2.4$ \\
\hline
B & 1.4 & 1.1 & 0 & 0.8 & $\geq 0.7$ \\
\hline
C & 2.3 & 5.6 & 11.1 & 1.3 & $\geq 5$ \\
\hline
\end{tabular}
\caption{Requisiti nutrizionali per la dieta}
\end{table}

\subsection{Formulazione matematica}

\subsubsection{Variabili decisionali}
\[ 
x_i \in \mathbb{R}_{\geq 0}, \quad \text{per } i = 1, \dots, n
\]

\subsubsection{Funzione obiettivo}
\[ 
\min \sum_{i=1}^{n} c_i x_i
\]

\subsubsection{Vincoli}
\[ 
\sum_{i=1}^{n} a_{ji} x_i \geq b_j,\quad j = 1, \dots, m,\qquad x_i \geq 0, \quad i = 1, \dots, n.
\]

L'istanza specifica è:
\begin{align}
\min \quad & 35x_1 + 50x_2 + 80x_3 + 95x_4 \\[2mm]
\text{soggetto a} \quad 
& 2.2x_1 + 3.4x_2 + 7.2x_3 + 1.5x_4 \geq 2.4 \quad && \text{(Nutriente A)} \\[1mm]
& 1.4x_1 + 1.1x_2 + 0.8x_4 \geq 0.7 \quad && \text{(Nutriente B)} \\[1mm]
& 2.3x_1 + 5.6x_2 + 11.1x_3 + 1.3x_4 \geq 5.0 \quad && \text{(Nutriente C)} \\[1mm]
& x_1, \, x_2, \, x_3, \, x_4 \geq 0.
\end{align}

\section{\textbf{\textit{Problema della miscelazione - Esempio 5}}}

\subsection{\textbf{\textit{Descrizione}}}
Produrre 100 quintali di acciaio al minimo costo utilizzando una combinazione di rottami metallici con specifiche chimiche:
\begin{itemize}
    \item Ferro: almeno 65\%
    \item Nichel: almeno 18\%
    \item Cromo: almeno 10\%
    \item Impurità: massimo 1\%
\end{itemize}

\subsection{Dati del problema}
\begin{table}[h]
\centering
\begin{tabular}{|l|c|c|c|c|c|c|}

\hline
\textbf{Componente} & \textbf{Rott. 1} & \textbf{Rott. 2} & \textbf{Rott. 3} & \textbf{Rott. 4} & \textbf{Rott. 5} & \textbf{Rott. 6} \\
\hline
Ferro [\%]       & 93  & 76  & 74  & 65  & 72  & 68  \\
\hline
Nichel [\%]      & 5   & 13  & 11  & 16  & 6   & 23  \\
\hline
Cromo [\%]       & 0   & 11  & 12  & 14  & 20  & 8   \\
\hline
Impurità [\%]    & 2   & 0   & 3   & 5   & 2   & 1   \\
\hline
Peso max [q]     & 30  & 90  & 50  & 70  & 60  & 50  \\
\hline
Costo [€/q]      & 50  & 100 & 80  & 85  & 92  & 115 \\
\hline
\end{tabular}
\caption{Composizione e dati dei rottami metallici}
\end{table}

\subsection{Formulazione matematica}

\subsubsection{Variabili decisionali}
Siano 
\[ 
x_i \in \mathbb{R}_{\geq 0}, \quad i=1,\dots,6,
\]
i quintali del rottame $i$-esimo utilizzati nella miscela.

\subsubsection{Funzione obiettivo}
\[ 
\min z = 50x_1 + 100x_2 + 80x_3 + 85x_4 + 92x_5 + 115x_6
\]

\subsubsection{Vincoli}
Quantità totale:
\[ 
x_1+x_2+x_3+x_4+x_5+x_6 = 100.
\]

Percentuale di ferro:
\[ 
93x_1+76x_2+74x_3+65x_4+72x_5+68x_6 \geq 65\cdot 100.
\]

Percentuale di nichel:
\[ 
5x_1+13x_2+11x_3+16x_4+6x_5+23x_6 \geq 18\cdot 100.
\]

Percentuale di cromo:
\[ 
0x_1+11x_2+12x_3+14x_4+20x_5+8x_6 \geq 10\cdot 100.
\]

Percentuale di impurità:
\[ 
2x_1+0x_2+3x_3+5x_4+2x_5+1x_6 \leq 1\cdot 100.
\]

Vincoli di disponibilità:
\[ 
\begin{aligned}
x_1 &\leq 30, \\
x_2 &\leq 90, \\
x_3 &\leq 50, \\
x_4 &\leq 70, \\
x_5 &\leq 60, \\
x_6 &\leq 50.
\end{aligned}
\]
 
\section{\textbf{\textit{Problema di miscelazione - Esempio 6}}}

\subsection{\textbf{\textit{Descrizione}}}
Un'azienda produce due tipi di fertilizzanti, A e B, mescolando due composti chimici; l'obiettivo è minimizzare il costo totale rispettando requisiti qualitativi e quantitativi:
\begin{itemize}
    \item Fertilizzante A $\geq$ 40000 kg (25\% azoto, 10\% ferro)
    \item Fertilizzante B $\geq$ 30000 kg (20\% azoto, 16\% ferro)
\end{itemize}

\subsection{Dati tecnologici}
\begin{table}[h]
\centering
\begin{tabular}{|l|c|c|c|c|}

\hline
\textbf{Composto} & \textbf{Azoto [\%]} & \textbf{Ferro [\%]} & \textbf{Quantità max [kg]} & \textbf{Costo [€/kg]} \\
\hline
Composto 1       & 50                 & 40               & 30000                    & 3                    \\
\hline
Composto 2       & 70                 & 6                & 25000                    & 4                    \\
\hline
\end{tabular}
\caption{Dati dei composti per la produzione dei fertilizzanti}
\end{table}

\subsection{Formulazione matematica}

\subsubsection{Variabili decisionali}
\begin{itemize}
    \item $x_{1A}$ : kg di composto 1 per il fertilizzante A
    \item $x_{2A}$ : kg di composto 2 per il fertilizzante A
    \item $x_{1B}$ : kg di composto 1 per il fertilizzante B
    \item $x_{2B}$ : kg di composto 2 per il fertilizzante B
\end{itemize}

\subsubsection{Funzione obiettivo}
\[ 
\min z = 3x_{1A}+4x_{2A}+3x_{1B}+4x_{2B}
\]

\subsubsection{Vincoli}
Requisiti di qualità per il fertilizzante A:
\[ 
\begin{aligned}
0.5x_{1A}+0.7x_{2A} &\geq 0.25(x_{1A}+x_{2A})\quad\Rightarrow\quad 0.5x_{1A}+0.7x_{2A}\geq 10000, \\
0.4x_{1A}+0.06x_{2A} &\geq 0.10(x_{1A}+x_{2A})\quad\Rightarrow\quad 0.4x_{1A}+0.06x_{2A}\geq 4000.
\end{aligned}
\]

Requisiti di qualità per il fertilizzante B:
\[ 
\begin{aligned}
0.5x_{1B}+0.7x_{2B} &\geq 0.20(x_{1B}+x_{2B})\quad\Rightarrow\quad 0.5x_{1B}+0.7x_{2B}\geq 0.20\cdot50000=10000, \\
0.4x_{1B}+0.06x_{2B} &\geq 0.16(x_{1B}+x_{2B})\quad\Rightarrow\quad 0.4x_{1B}+0.06x_{2B}\geq 8000.
\end{aligned}
\]

Vincoli sulla disponibilità dei composti:
\[ 
x_{1A}+x_{1B}\leq 30000,\quad x_{2A}+x_{2B}\leq 25000.
\]

Vincoli di non negatività:
\[ 
x_{1A},\, x_{2A},\, x_{1B},\, x_{2B}\geq 0.
\]

\subsection{Formulazione generale}
\[ 
\begin{aligned}
\min \quad & 3x_{1A}+4x_{2A}+3x_{1B}+4x_{2B} \\
\text{soggetto a}\quad 
& 0.5x_{1A}+0.7x_{2A}\geq 10000, \\
& 0.4x_{1A}+0.06x_{2A}\geq 4000, \\
& 0.5x_{1B}+0.7x_{2B}\geq 10000, \\
& 0.4x_{1B}+0.06x_{2B}\geq 8000, \\
& x_{1A}+x_{1B}\leq 30000, \\
& x_{2A}+x_{2B}\leq 25000, \\
& x_{1A},\, x_{2A},\, x_{1B},\, x_{2B}\geq 0.
\end{aligned}
\]

\section{Modelli Multi-periodo per la pianificazione della produzione}
\subsection{Descrizione del problema}

Consideriamo un orizzonte temporale suddiviso in $T$ periodi, indicati con $t=1,\ldots,T$ in ciascun periodo:
\begin{itemize}
    \item La domanda di ciascun prodotto è nota con $d_t$
    \item Il costo unitario di produzione è $c_t$
    \item È possibile stoccare il prodotto non venduto, con un costo di stoccaggio unitario $h_t$ (che può variare nel tempo)
\end{itemize}

\subsection{Obiettivo}
L'obiettivo è determinare quanto produrre in ciascun periodo per soddisfare la domanda e minimizzare i costi totali di produzione e stoccaggio.

\subsection{Variabili decisionali}
\begin{itemize}
    \item $x_t$ : quantità prodotta nel periodo $t$
    \item $I_t$ : quantità stoccata alla fine del periodo $t$
\end{itemize}

\subsection{Equazioni di bilanciamento}
La quantità disponibile in ogni periodo è data dalla somma tra lo stock iniziale e la produzione. 
Questa quantità deve coprire la domanda e lasciare eventualmente un residuo da stoccare:
\[ 
I_{t-1} + x_t = d_t + I_t, \quad \text{per } t = 1, \dots, T
\]

\begin{itemize}
    \item $I_0$ : Livello iniziale di scorte (dato)
    \item In alcuni casi anche $I_T$ è fissato (ad esempio $I_T=0$ per evitare scorte residue alla fine)
\end{itemize}

\subsection{Funzione obiettivo}
Minimizzare il costo totale di produzione e stoccaggio:
\[ 
\min \sum_{t=1}^{T} \left( c_t x_t + h_t I_t \right)
\]

\subsection{Vincoli}

Equazioni di bilanciamento:
\[ 
I_{t-1} + x_t = d_t + I_t, \quad \text{per } t = 1, \dots, T
\]

Vincoli di non negatività:
\[ 
x_t \geq 0, \quad I_t \geq 0, \quad \text{per } t = 1, \dots, T
\]

(Opzionali) Vincoli di capacità produttiva:
\begin{itemize}
    \item Capacità massima di produzione in ogni periodo $x_t \leq u_t$
    \item Capacità massima di stoccaggio in ogni periodo $I_t \leq S_t$
\end{itemize}

\[ 
x_t \leq X_t, \quad I_t \leq S_t, \quad \text{per alcuni } t \in \{1, \dots, T\}
\]

\subsection{Formulazione generale}
\[ 
\begin{aligned}
\min \quad & \sum_{t=1}^{T} \left( c_t x_t + h_t I_t \right) \\
\text{soggetto a} \quad 
& I_{t-1} + x_t = d_t + I_t, \quad t = 1, \dots, T \\
& x_t \geq 0, \quad I_t \geq 0, \quad t = 1, \dots, T \\
& x_t \leq u_t, \quad I_t \leq S_t, \quad \text{(se applicabile)}
\end{aligned}
\]

\section{Modelli Multi-Periodo - Esempio 7}

\subsection{Descrizione del problema}

Un'azienda riceve una commessa per la produzione di lamierati di zinco 
con specifiche dimensioni. Le consegne devono avvenire alla fine di ogni mese, 
senza possibilità di posticipo. Il processo produttivo è gestito da un 
solo stabilimento, e le quantità richieste sono note per ciascun periodo.

\subsection{Dati del problema}

\subsection{Dati aggiuntivi}
\begin{itemize}
    \item Scorte iniziali: $I_0 = 100$ unità
    \item Periodi di pianificazione: $t = 1, 2, 3$
    \item La domanda in ciascun periodo deve essere soddisfatta esattamente
\end{itemize}

\subsection{Dati tecnologici}
\begin{table}[h]
\centering
\begin{tabular}{|c|c|c|c|c|}

\hline
Periodo & Domanda $d_t$ & Capacità $C_t$ & Costo produzione $c_t$ [€/unità] & Costo stoccaggio $h_t$ [€/unità] \\
\hline
1 & 270 & 250 & 12 & 1.2 \\
\hline
2 & 290 & 220 & 14 & 1.1 \\
\hline
3 & 250 & 280 & 16 & 0.9 \\
\hline
\end{tabular}
\caption{Dati tecnologici per la pianificazione multi-periodo}
\end{table}

\subsection{Formulazione matematica}
\subsubsection{Variabili decisionali}

Siano:
\begin{itemize}
    \item $x_t$ : quantità prodotta nel periodo $t$
    \item $I_t$ : quantità stoccata alla fine del periodo $t$
\end{itemize}

\subsubsection{Funzione obiettivo}
\[ 
\min z(x, I) = 12x_1 + 14x_2 + 16x_3 + 1.2I_1 + 1.1I_2 + 0.9I_3
\]

\subsubsection{Vincoli}
Equazioni di bilanciamento (derivate da $I_{t-1}+x_t=d_t+I_t$):
\[ 
\begin{aligned}
x_1 - I_1 &= 170 \\
x_2 + I_1 - I_2 &= 290 \\
x_3 + I_2 - I_3 &= 250
\end{aligned}
\]

Queste equazioni derivano dalla formula di bilancio del magazzino $I_{t-1}+x_t=d_t+I_t$
considerando la scorta iniziale $I_0 = 100$:
\begin{align*}
I_0 + x_1 &= d_1 + I_1 \Rightarrow 100 + x_1 = 270 + I_1 \Rightarrow x_1 - I_1 = 170 \\
I_1 + x_2 &= d_2 + I_2 \Rightarrow I_1 + x_2 = 290 + I_2 \Rightarrow x_2 + I_1 - I_2 = 290 \\
I_2 + x_3 &= d_3 + I_3 \Rightarrow I_2 + x_3 = 250 + I_3 \Rightarrow x_3 + I_2 - I_3 = 250
\end{align*}

Vincoli di capacità produttiva:
\[ 
\begin{aligned}
x_1 &\leq 250 \\
x_2 &\leq 220 \\
x_3 &\leq 280
\end{aligned}
\]

Vincoli di non negatività:
\[ 
x_t \geq 0, \quad I_t \geq 0, \quad \text{per } t = 1, 2, 3
\]

\subsection{Formulazione generale}
\[ 
\begin{aligned}
\min \quad & 12x_1 + 14x_2 + 16x_3 + 1.2I_1 + 1.1I_2 + 0.9I_3 \\
\text{soggetto a} \quad 
& x_1 - I_1 = 170 \\
& x_2 + I_1 - I_2 = 290 \\
& x_3 + I_2 - I_3 = 250 \\
& x_1 \leq 250 \\
& x_2 \leq 220 \\
& x_3 \leq 280 \\
& x_1, x_2, x_3, I_1, I_2, I_3 \geq 0
\end{aligned}
\]

\section{Problema del trasporto}
\subsection{Descrizione del problema}
Il \textbf{\textit{problema del trasporto}} è un classico modello di \textbf{\textit{ottimizzazione lineare}} 
che riguarda la \textbf{\textit{distribuzione ottimale di beni}} da un insieme di fornitori 
a un insieme di clienti, \textbf{\textit{minimizzando i costi di trasporto}}.

\subsection{Dati del problema}
\begin{itemize}
    \item $n$ \textbf{\textit{fornitori}}, ciascuno con una capacità massima $a_i$
    \item $m$ \textbf{\textit{clienti}}, ciascuno con una domanda minima $b_j$
    \item $c_{ij}$ come \textbf{\textit{costo di trasporto unitario}} da ogni fornitore a ogni cliente
\end{itemize}

\subsection{Ipotesi di ammissibilità}
Il problema ammette una soluzione solo se la \textbf{\textit{quantità totale disponibile}} 
è almeno pari alla \textbf{\textit{domanda totale}}:
\[
\underline{\sum_{i=1}^{n} a_i \geq \sum_{j=1}^{m} b_j}
\]

Se tutti i fornitori sono collegati a tutti i clienti, questa condizione 
è sufficiente per garantire l'esistenza di una soluzione ammissibile.

\subsection{Formulazione matematica}
\subsubsection{Variabili decisionali}
Siano $x_{ij}$ le \textbf{\textit{quantità da trasportare}} dal fornitore $i$ al cliente $j$.
\[
x_{ij} \in \mathbb{R}_{\geq 0}, \quad \text{per } i = 1, \dots, n;\ j = 1, \dots, m
\]

\subsubsection{Funzione obiettivo}
\textbf{\textit{Minimizzare il costo totale di trasporto}}:
\[
\min \sum_{i=1}^{n} \sum_{j=1}^{m} c_{ij} x_{ij}
\]

\subsubsection{Vincoli}
\textbf{\textit{Capacità dei fornitori}}: ogni fornitore non può spedire più della propria disponibilità:
\[
\sum_{j=1}^{m} x_{ij} \leq a_i, \quad \text{per } i = 1, \dots, n
\]

\textbf{\textit{Domanda dei clienti}}: ogni cliente deve ricevere almeno la quantità richiesta:
\[
\sum_{i=1}^{n} x_{ij} \geq b_j, \quad \text{per } j = 1, \dots, m
\]

\textbf{\textit{Non negatività}}:
\[
x_{ij} \geq 0, \quad \text{per } i = 1, \dots, n;\ j = 1, \dots, m
\]

\subsection{Formulazione generale}
\[
\begin{aligned}
\min \quad & \sum_{i=1}^{n} \sum_{j=1}^{m} c_{ij} x_{ij} \\
\text{soggetto a} \quad 
& \sum_{j=1}^{m} x_{ij} \leq a_i, \quad i = 1, \dots, n \\
& \sum_{i=1}^{n} x_{ij} \geq b_j, \quad j = 1, \dots, m \\
& x_{ij} \geq 0, \quad i = 1, \dots, n;\ j = 1, \dots, m
\end{aligned}
\]

\subsection{Difficoltà del problema}
Ci sono diverse tipologie di difficoltà:
\begin{itemize}
    \item Capacità massime per gli archi
    \item Non tutti i fornitori sono collegati a tutti i destinatari
    \item Il trasporto non avviene direttamente tra fornitore e destinatario ma attraverso dei centri di raccolta e distribuzione intermedi
    \item Possibili perdite nel trasporto
    \item Più prodotti da trasportare con vincoli di capacità comuni
\end{itemize}

\section{Schedulazione delle attività}
\subsection{Descrizione del problema}

Il problema di schedulazione delle attività consiste nel pianificare un insieme di attività,
tenendo conto di:
\begin{itemize}
    \item Vincoli di precedenza: alcune attività non possono iniziare finché altre non sono terminate
    \item Minimizzazione del tempo totale di completamento (makespan): si vuole completare tutte le attività nel minor tempo possibile
\end{itemize}

\subsection{Dati del problema}
Come dicevamo, l'obiettivo è minimizzare il makespan, che è il tempo totale necessario per completare tutte le attività.

I dati noti sono:
\begin{itemize}
    \item $p_i$: durata dell'attività $i$
    \item $prec(i)$: insieme delle attività che devono essere completate prima dell'attività $i$
\end{itemize}

Le variabili decisionali sono:
\begin{itemize}
    \item $t_i \in \mathbb{R}$: tempo di inizio dell'attività $i$, per $i=1,\ldots,n$
\end{itemize}

Nella rappresentazione grafica sono utili i concetti di:
\begin{itemize}
    \item Nodi: rappresentano eventi, ovvero l'inizio di un'attività o la fine dell'ultima attività precedente
    \item Archi: rappresentano le attività stesse, con la loro durata associata
\end{itemize}

\subsection{Formulazione matematica}
\subsubsection{Funzione obiettivo}
Minimizzare il makespan:
\[ 
\min \max_{i=1,\dots,n} \{ t_i + p_i \}
\]

\subsubsection{Vincoli}
Vincoli di precedenza: un'attività può iniziare solo dopo che tutte le sue attività precedenti sono state completate:
\[ 
t_i \geq t_j + p_j \quad \forall i = 1,\dots,n,\ \forall j \in \text{prec}(i)
\]

Vincoli di non negatività: il tempo di inizio di ogni attività deve essere maggiore o uguale a zero:
\[ 
t_i \geq 0 \quad i = 1,\dots,n
\]

\subsubsection{Linearizzazione del problema}
Per rendere il problema più trattabile computazionalmente, possiamo introdurre una variabile ausiliaria $T$ che rappresenta il makespan:
\[ 
\min T
\]

Soggetto a:
\[ 
\begin{aligned}
& T \geq t_i + p_i \quad \forall i = 1,\dots,n \\
& t_i \geq t_j + p_j \quad \forall i = 1,\dots,n,\ \forall j \in \text{prec}(i) \\
& t_i \geq 0 \quad \forall i = 1,\dots,n
\end{aligned}
\]

N.B. Il vincolo $T \geq t_i + p_i$ può essere applicato solo alle attività 
senza successori, poiché solo queste determinano effettivamente il makespan.

\subsection{Osservazioni importanti}
Nei problemi reali di produzione è necessario includere vincoli aggiuntivi, come:
\begin{itemize}
    \item Conflitti di risorse: due attività non possono essere eseguite contemporaneamente se richiedono la stessa macchina o risorsa
    \item Questi problemi sono molto piò complessi e spesso richiedono programmazione lineare mista (MILP), anche se non è sempre conveniente risolverli con questo approccio
\end{itemize}

La soluzione del problema fornisce anche:
\begin{itemize}
    \item Attività critiche, cioè quelle che non possono essere ritardate senza influenzare il makespan
    \item Slack time, cioè il tempo di flessibilità per ogni attività
\end{itemize}

\subsection{Earliest Starting Time (EST)}

L'obiettivo è trovare il tempo di inizio al più presto per ogni attività, 
minimizzando la somma dei tempi di inizio delle attività.

Il modello di programmazione lineare è il seguente:
\[ 
\min \sum_{i=1}^n t_i
\]

Nei vincoli abbiamo:
\begin{itemize}
    \item Tempo di completamento del progetto:
    \[T = T*\]
    \item Ogni attività deve finire prima di T:
    \[t_i + p_i \leq T, \quad \forall i = 1,\ldots,n\]
    \item Ogni attività deve iniziare dopo le sue precedenti:
    \[t_i \geq t_j + p_j, \quad \forall i = 1,\ldots,n,\ \forall j \in \text{prec}(i)\]
    \item Non negatività:
    \[t_i \geq 0, \quad \forall i = 1,\ldots,n\]
\end{itemize}

\subsection{Latest Starting Time (LST)}
L'obiettivo è trovare il tempo di inizio al più tardi per ogni attività, 
massimizzando la somma dei tempi di inizio delle attività senza ritardare il 
completamento del progetto.

Il modello di programmazione lineare è il seguente:
\[ 
\max \sum_{i=1}^n t_i
\]

Nei vincoli abbiamo:
\begin{itemize}
    \item Tempo di completamento del progetto:
    \[T = T*\]
    \item Ogni attività deve finire prima di T:
    \[t_i + p_i \leq T, \quad \forall i = 1,\ldots,n\]
    \item Ogni attività deve iniziare dopo le sue precedenti:
    \[t_i \geq t_j + p_j, \quad \forall i = 1,\ldots,n,\ \forall j \in \text{prec}(i)\]
    \item Non negatività:
    \[t_i \geq 0, \quad \forall i = 1,\ldots,n\]
\end{itemize}

\subsection{Attività critiche}
Un'attività è critica se:
\[ 
EST_i = LST_i
\]

In altre parole, se il tempo di inizio più presto coincide con il tempo di inizio più tardi,
significa che non c'è flessibilità e qualsiasi ritardo in questa attività influenzerà il completamento del progetto.