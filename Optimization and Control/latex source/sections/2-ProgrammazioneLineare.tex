% filepath: c:\Users\mrheltic\Documents\GitHub\dispense\Optimization and Control\latex source\sections\1-Introduzione.tex

\chapter{Programmazione Lineare}

\section{Problemi e Istanze}
Un problema P è una domanda o una questione la cui risposta dipende dal valore assunto da un insieme di parametri e variabili.

Un problema è formalmente definito da:
\begin{itemize}
    \item Descrizione dei parametri, ovvero le quantità note che caratterizzano il problema
    \item Descrizione delle proprietà che deve soddisfare la soluzione cercata
\end{itemize}

Una formulazione comune di un problema prevede:
\begin{itemize}
    \item L'insieme delle soluzioni ammissibili F, cioè tutte le possibili risposte che rispettano i vincoli imposti
    \item Una funzione obiettivo z che assegna un valore (costo, profitto, ecc.) a ciascuna soluzione
\end{itemize}

Formalmente:
\begin{equation}
z \colon F \rightarrow \mathbb{R}
\end{equation}

Un'istanza di un problema P, invece, è una versione specifica del problema ottenuta fissando i valori dei parametri. In altre parole, è un caso concreto del problema generale.

\section{Problemi di Ottimizzazione}
Un problema di ottimizzazione è un problema in cui si cerca, tra tutte le soluzioni ammissibili $x \in F$, una soluzione ottima $x^*$ che massimizza o minimizza la funzione obiettivo $z$.

Formulazioni equivalenti:
\begin{itemize}
    \item Minimizzazione
    \begin{equation}
    P = (F, z; \min) \Leftrightarrow \min\{z(x) : x \in F\}
    \end{equation}
    
    \item Massimizzazione
    \begin{equation}
    P = (F, z; \max) \Leftrightarrow \max\{z(x) : x \in F\}
    \end{equation}
\end{itemize}

\section{Problemi di programmazione Lineare}
Un problema di programmazione lineare è un caso particolare di problema di ottimizzazione in cui:
\begin{itemize}
    \item La funzione obiettivo è lineare
    \item I vincoli che definiscono l'insieme F sono lineari e chiusi (cioè includono i punti di frontiera)
    \item Le variabili sono continue (non intere)
    \item I parametri sono deterministici (noti e non soggetti a incertezza)
\end{itemize}

La sua forma esplicita è del tipo:
\begin{align}
\max & \quad c_1x_1 + c_2x_2 + \cdots + c_nx_n \\
\text{soggetto a} & \quad a_{11}x_1 + a_{12}x_2 + \cdots + a_{1n}x_n \leq b_1 \\
& \quad \vdots \\
& \quad a_{m1}x_1 + a_{m2}x_2 + \cdots + a_{mn}x_n \leq b_m \\
& \quad x_1, x_2, \ldots, x_n \geq 0
\end{align}

È possibile scriverla anche in forma matriciale, ovvero:
\begin{align}
\max & \quad z = c^T x \\
\text{soggetto a} & \quad Ax \leq b \\
& \quad x \geq 0
\end{align}

Dove:
\begin{itemize}
    \item $c \in \mathbb{R}^n$ : vettore dei coefficienti della funzione obiettivo;
    \item $A \in \mathbb{R}^{m \times n}$ : matrice dei coefficienti dei vincoli (matrice tecnologica);
    \item $b \in \mathbb{R}^m$ : vettore dei termini noti;
    \item $x \in \mathbb{R}^n$ : vettore delle variabili decisionali;
    \item $F = \{x \in \mathbb{R}^n : Ax \leq b, x \geq 0\}$ : insieme delle soluzioni ammissibili.
\end{itemize}

\section{Esempio di pianificazione della produzione - 1}

\subsection{Descrizione}
L'obiettivo è massimizzare il profitto derivante dalla produzione e vendita di due tipi di biocarburanti, quello di tipo 1 e quello di tipo 2.

\subsection{Vincoli}
\begin{itemize}
    \item Produzione divisa in tre stabilimenti: uno per la preparazione, uno per la purificazione e uno per l'estrazione
    \item Ogni stabilimento ha una capacità giornaliera limitata (in termini di ore)
    \item Entrambi i biocarburanti devono essere lavorati in ciascuno dei tre stabilimenti
\end{itemize}

\subsection{Dati tecnologici}
\begin{table}[h]
\centering
\resizebox{\textwidth}{!}{
\begin{tabular}{|c|c|c|c|}
\hline
\textbf{Stabilimento} & \textbf{Biocarburante 1 (h/ton)} & \textbf{Biocarburante 2 (h/ton)} & \textbf{Capacità giornaliera (h)} \\
\hline
Preparazione   & 0.72 & 0.85 & 18 \\
\hline
Purificazione  & 1.68 & 1.42 & 18 \\
\hline
Estrazione     & 1.92 & 2.12 & 16 \\
\hline
\end{tabular}
}
\end{table}

\subsection{Formulazione matematica}

\subsubsection{Variabili decisionali}
Siano:
\begin{itemize}
    \item $x_1$ : quantità (in tonnellate) di Biocarburante 1 da produrre
    \item $x_2$ : quantità (in tonnellate) di Biocarburante 2 da produrre
\end{itemize}

Le variabili sono continue e non negative:
\begin{equation}
x_1, x_2 \in \mathbb{R}_{\geq 0}
\end{equation}

\subsubsection{Funzione obiettivo}
Massimizzare il profitto totale:
\begin{equation}
\max z(x) = 540x_1 + 590x_2
\end{equation}

\subsubsection{Vincoli di capacità}

\[
\begin{cases}
0.72x_1 + 0.85x_2 \leq 18 \quad &\text{(Preparazione)} \\
1.68x_1 + 1.42x_2 \leq 18 \quad &\text{(Purificazione)} \\
1.92x_1 + 2.12x_2 \leq 16 \quad &\text{(Estrazione)} \\
x_1, x_2 \geq 0 \quad &\text{(Non negatività)}
\end{cases}
\]


\section{Esempio di pianificazione della produzione - 2}

\subsection{Descrizione}
L'obiettivo è massimizzare il profitto settimanale, calcolato come differenza tra ricavi di vendita e costi di produzione, per tre tipi di argilla: A, B e C.

\subsection{Vincoli}
\begin{itemize}
    \item La produzione avviene in tre stabilimenti, ciascuno con una capacità settimanale limitata (in ore)
    \item Ogni tipo di argilla può essere lavorato in uno qualsiasi dei tre stabilimenti
    \item La quantità totale di argilla A deve essere compresa tra il 50\% e il 70\% della produzione complessiva
\end{itemize}

\subsection{Dati tecnologici}
\begin{table}[h]
\centering
\begin{tabular}{|c|c|c|c|c|c|}
\hline
\textbf{Stabilimento} & \textbf{Argilla A} & \textbf{Argilla B} & \textbf{Argilla C} & \textbf{Capacità} & \textbf{Costo} \\
& \textbf{18€/quintale} & \textbf{21€/quintale} & \textbf{24€/quintale} & \textbf{settimanale (h)} & \textbf{[€/h]} \\
\hline
1 & 0.18 & 0.21 & 0.24 & 90 & 3.52 \\
\hline
2 & 0.20 & 0.18 & 0.21 & 85 & 4.18 \\
\hline
3 & 0.12 & 0.22 & 0.23 & 80 & 3.98 \\
\hline
\end{tabular}
\end{table}

\subsection{Formulazione matematica}

\subsubsection{Variabili decisionali}
Siano:
\begin{itemize}
    \item $x_{iA}$ : quantità di argilla A prodotta nello stabilimento $i$
    \item $x_{iB}$ : quantità di argilla B prodotta nello stabilimento $i$
    \item $x_{iC}$ : quantità di argilla C prodotta nello stabilimento $i$
\end{itemize}

Quindi, siano:
\begin{equation}
x_{iA}, x_{iB}, x_{iC} \in \mathbb{R}_{\geq 0}, \quad \text{per } i=1,2,3
\end{equation}

\subsubsection{Funzione obiettivo}
Massimizzare il profitto, ovvero la differenza tra ricavi e costi, cioè:
\begin{align*}
\max z = & \ 18(x_{1A} + x_{2A} + x_{3A}) + 21(x_{1B} + x_{2B} + x_{3B}) + 24(x_{1C} + x_{2C} + x_{3C}) \\
& - \left[ 3.52(0.18x_{1A} + 0.21x_{1B} + 0.24x_{1C}) \right. \\
& \left. + 4.18(0.20x_{2A} + 0.18x_{2B} + 0.21x_{2C}) \right. \\
& \left. + 3.98(0.12x_{3A} + 0.22x_{3B} + 0.23x_{3C}) \right]
\end{align*}

\subsubsection{Vincoli di capacità}
\[
\begin{cases}
    0.18x_{1A}+0.21x_{1B}+0.24x_{1C} \leq 90, \\
    0.20x_{2A}+0.18x_{2B}+0.21x_{2C} \leq 85, \\
    0.12x_{3A}+0.22x_{3B}+0.23x_{3C} \leq 80
\end{cases}
\]

\subsubsection{Vincoli di proporzione sulla produzione di argilla A}
\begin{align}
x_{1A}+x_{2A}+x_{3A} &\geq 0.5 \cdot T \\
x_{1A}+x_{2A}+x_{3A} &\leq 0.7 \cdot T
\end{align}

Dove: $T = \sum_{i=1}^{3}(x_{iA}+x_{iB}+x_{iC})$